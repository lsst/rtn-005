\section{Introduction}

This document provides a guide to the \VRO  approach to project management.
% PJM: \VRO macro is both nadly named and missing a \xspace at the end of it.
See also the the operations proposal \citeds{LDO-31}
In operations, there is no \gls{Primavera} or \gls{EVMS} as there are for the Rubin Construction Project.

There are a few main ways work will be managed, all within an Agile framework. The Agile method is a set of practices that allow the ability to adapt and respond to change through collaboration between self-managing, cross-functional teams.  We focus on the people doing the work and how they work together, instead of organizing the program into functional silos.

This framework allows the multidisciplinary Rubin team to operate the facility and generate nightly data products while continuously improving efficiency of workflows ('kanban' style), as well as iteratively responding to user feedback on a longer timescale to maximize the scientific benefit of annual data releases ('scrum' style).

Some examples that will use the kanban style are the daily functionality checks needed for nightly operations of the telescope or the regular monitoring of progress toward achieving milestones and reporting to governing organizations.
For either of these examples, we will maintain a set of tasks that need to be performed in sequence every cycle.
Each task could be described individually inside a JIRA ticket and grouped together in a set, so that any person doing the daily checks, for example, could be assigned the set of tasks, open the tickets, comment on results, pass additional work onto others as needed, and mark the tickets done as the work is completed.
In this way, the completion of the checklist of tasks is transparent and traceable and can be monitored, and the efficiency of the steps in the flow of work can be improved upon continuously.

Examples of scrum-like work include improving algorithms in response to user community feedback, optimizing the observing strategy as the survey progresses, and other incremental work needed to produce the annual data releases.

In this document, we lay out the procedural details for how we define and carry out plans, following the framework described above, to deliver on our milestones, maintain visibility in our workflows, remain responsive to change, and offer staff the ability to innovate and collaborate.


\section{Useful Contacts}
\label{sec:contacts}

The \RO Acting \gls{Director}  is Robert Blum, the Interim Deputy Director for NOIRLab is Amanda Bauer, and the Deputy Director for SLAC is Phil Marshall.
% PJM: \RO macro needs a trailing \xspace
They are the first point of contact for all issues regarding project management within \RO.


The  Project Controls Specialist is Cathy Petry.
She monitors the budgets and maintains details within WeBud, the NOIRLab budgeting system. She assists in tracking milestones and reporting.

\section{Associate Directors}
\label{sec:tcam}

Rubin Operations has four operational Departments and the Director’s Office: RDO (Rubin Director’s Office), ROO (Rubin Observatory Operations), RDP (Rubin Data Production), RSP (Rubin System Performance), and REO (Rubin Education and Public Outreach). Each operational Department is lead by an Associate Director.

\section{Formal Organizational Structure}
\label{sec:structure}

\subsection{Work Breakdown Structure}
\label{sec:wbs}


The \gls{WBS} provides a hierarchical index of all hardware, software, services, and other deliverables which are required to operate \RO.
Thus:

% PJM: The WBS Activities table down to level 3 is a very big table. Do we need the table below to name the dept/team/group as well as the description of the activity? I think we can introduce these items in teh table even though they are not referred to until the next subsection.
\begin{longtable}[]{@{}lll@{}}
\hline
\gls{WBS} & Description & Lead\tabularnewline
\hline
\endhead
NUM & DESC & Lead\tabularnewline
\hline
\end{longtable}

These subdivisions are referred to as the \emph{third level \gls{WBS}}.
Often, they are quoted without the leading ``1'' (e.g. ``02C.01''), but, even in this form, they are referred to as ``third level''.

Nodes in the \gls{WBS} tree are referred to as \glspl{element}.
All of these third level \gls{WBS} \glspl{element}s (activities) are subdivided, forming a fourth level.
The fourth level elements' numbers always contains a ``00'' \gls{element}, which is used to capture management and \gls{LOE} work, and may contain other fourth level, or even deeper, structure.

\subsection{Organization Breakdown
Structure}\label{organization-breakdown-structure}

%PJM: I guess this is where we explain how the WBS activities map cleanly on to the depts, teams, and groups. IIUC, level 0 is the observatory, level 1 is the departments, level 2 is the teams, level 3 is the groups within teams.

\subsection{The Control Account Manager}
\label{sec:cam}

%PJM: The original text below described how the Construction Project is organized. In operations we do not have CAMs (yet): we have ADs and Team Leaders. See proposed changes!

% OLD:
% A \gls{CA} is the intersection between the \gls{WBS} and the \gls{OBS}.
% Each \gls{CA} falls under the purview of a \gls{CAM}.
% Typically within Data Production, a single \gls{CAM} is responsible for the whole of a third level \gls{WBS}.
% That is, the manager at the lead institution for a particular component is responsible for all work performed on that \gls{WBS} \gls{element}, even if some of that work is performed at another institution.

% Proposed:
In general, a \gls{CA} is the intersection between the \gls{WBS} and the \gls{OBS}.
Each \gls{CA} falls under the purview of a \gls{CAM}.
At \RO, a single Team Leader is responsible for the whole of a third level \gls{WBS}, and therefore acts as the \gls{CAM}.
That is, the Team Leader is the manager for that particular \gls{WBS} \gls{element}, and is responsible for all work performed on it, even though the work may be performed by staff outside the Team Leader's institution.


\subsection{Level of Effort Work}
\label{sec:loe}

There is work throughout Rubin Operations that will be recorded as a \gls{LOE}. These activities include attending meetings, reporting on milestones or taking part in other activities which do not directly map to deployed code.
This may be particularly the case for technical managers or others in leadership roles within the project.
With \gls{LOE} work is assumed to earn value (informally) simply through the passage of time.

In general, we strive to minimize the fraction of our effort which is devoted to \gls{LOE} activities and favor those which are more directly accountable.
However, in certain cases such as operations of pipelines or other systems, \gls{LOE} is perfectly acceptable.

As an example, our first-order estimate is that developers in Data Production will spent 30\% of their time on \gls{LOE} type activities, and the remaining 70\% of their effort is tracked against concrete deliverables.
However, as above, we generally aspire to minimize the fraction of \gls{LOE}: Team Leaders (CAMs) are therefore encouraged to explicitly schedule as much developer time as is possible.

Note that all \gls{LOE} work should be invoiced to the ``00'' fourth-level \gls{WBS} \gls{element} (1.02C.03.00, 1.02C.04.00, etc)

\section{Estimating Effort}
\label{sec:effort}

\subsection{Basic Assumptions}
The Construction Project assumes that a full-time individual works for a total of 1,800 hours per year: this figure is \emph{after} all vacations, sick leave, etc are taken into account.
The Rubin operations partners, SLAC and NOIRLab, may have different definitions for tracking their staff time; Rubin operations uses 1800 hours per year as a fiducial value for effort estimation purposes.

% PJM: generalize to ops staff
% Staff appointed to ``developer''
% positions are expected to devote this effort directly to \gls{LSST}.
In general, staff in Rubin operations roles at a given expected full-time equivalent (FTE) effort level are expected to devote that fraction of their total work time to \RO.

Staff in ``scientist'' roles are expected to spend 20\% of their time on personal research (see the Rubin Operations Plan for details).
That is, scientists are expected to devote 1,440 hours per
year to operations activity, and the remainder of their time to personal research.

% PJM: I don't think this is how ops works at all:
% Personal research time is \emph{not} chargeable to \gls{LSST} under any \gls{WBS} or account, including level of effort.
% The Project expects to pay the full rate for an individual with research time who contributes 1,440 hours to the project, and does not require any accounting of the remaining 360 hours.
% Instead:
Personal research time is charged to Rubin along with the operations role time, the logic being that scientists in operations roles must be engaged in research as well in order to succeed in their operations role.

When reporting actual costs (\S\ref{sec:actuals}), it may
be helpful to consider the following examples:

A developer, engineer or technician for which the total annual cost (salary, overheads, etc) is
\(A\) charges an hourly rate of \(A / 1800\).

% A scientist with total annual cost \(\gls{B}\) charges an hourly rate of
\(\gls{B} / 1440\).
A scientist with total annual cost \(\gls{B}\) ALSO charges an hourly rate of \(\gls{B} / 180\).

No further corrections are necessary. In particular, there is no
difference in the way working hours are measured, or the conversion of
\glspl{SP} to hours.
(See below for more discussion of story points.)

In Data Production, our base assumption is that 30\% of an individual's \RO operations time (i.e. 540 hours/year for a full-time developer, 432 hours/year for a full-time scientist) are devoted to overhead for regular meetings\footnote{``Meetings'' include, for example, scheduled weekly team meetings, stand-ups, etc; major conferences or project meetings involving preparation, travel time, etc should be scheduled in advance and allocated \glspl{SP}.}, ad-hoc discussions and other interruptions.
This work is counted as \gls{LOE} (and, as such, is charged to the relevant ``00'' fourth level \gls{WBS} \gls{element}, as described in \S\ref{sec:loe}).
It is actively encouraged to allocate less than 30\% of an individuals time to \gls{LOE} where that is possible.

Assuming no variation throughout the year, we therefore expect 105 hours of productive work from a developer, or 84 hours from a scientist, per month.
Note that this is averaged across the year: some months, such as those containing major holidays, will naturally involve less working time than others: the remainder will necessarily include more working time to compensate.

% PJM: "Jira" needs a macro, so we always use either "Jira" or "JIRA" but not both.
Rather than working in hours, our Jira based system uses Story Points (\gls{SP}), with one \gls{SP} being defined as equivalent to four hours of effort (half a day's work) by a competent developer.
Thus, we expect developers and scientists to produce 26.25 and 21 \glspl{SP} per \emph{average} month respectively.

\begin{table}
\begin{longtable}[]{@{}lrrr@{}}
\hline
          & \multicolumn{2}{c}{Hours} & \multicolumn{1}{c}{\glspl{SP}} \\
          & Per year & Per month      & Per month \\
\hline
Full-time Developer & 1800     & 105            & 26.25 \\
Full-time Scientist & 1440     &  84            & 21.00 \\
\hline
\end{longtable}
\caption{Expected working rates for developers and scientists. Technicians and engineers follow the same rates as developers.}
\label{tab:working-rate}
\end{table}

\subsection{Special Cases}

\subsubsection{Newcomers}
\label{sec:newcomers}

New or inexperienced developers, even when devoting their full attention to \gls{story}-pointed work, will likely be less productive than their more experienced peers.
In this case, the ratio of hours to \glspl{SP} increases, but the number of hours remains constant.

Note that specific activities related on ``onboarding'' and getting up to speed with the project can be ticketed as regular work.
For example, working through tutorials, reading documentation, and so on are all activities which can earn \glspl{SP}.
Typically, this will assigned to the ``00'' (management) \gls{element} of the \gls{WBS} (\S\ref{sec:wbs}).

\subsubsection{Technical Managers and other Leadership Roles}

Individuals in leadership roles may find it necessary to assign a larger fraction of their time to \gls{LOE} type work, and therefore spend fewer hours generating \glspl{SP}.
The ratio of hours to \glspl{SP} remains constant, but the number of hours decreases.

\section{Long Term Planning}
\label{sec:long-term-plan}

The authoritative summary of the long-term planning system may be found in ...
Here we expand upon the details of that system.
The plan for pre-operations and Survey Operations is embodied in:

\begin{enumerate}
\item
  A series of \emph{epics}, which describe major pieces of
  technical work. Epics are associated with concrete, albeit
  high-level, deliverables (in the \gls{shape} of milestones, below), and have
  specific resource loads (staff assignments), start dates, and
  durations.
\item
  \emph{Milestones} represent the delivery or availability of specific
  functionality. Each planning package culminates in a milestone, and
  may contain other milestones describing intermediate results.
\end{enumerate}


Milestones are allocated to one of four levels, defined as follows:

\begin{description}
\item[Level 1]
These are at the full observatory level owned by the directors office
\item[Level 2]
These reflect cross department commitments. As such, they must be defined
in consultation with the Director's Office
\item[Level 3]
These are internal to a particular Department and assigned to a team  and can therefore be specified by a single team lead.
\end{description}

Some of these are exposed to external reviewers: it is vital that these
be delivered on time and to specification. Low-level milestones are
defined for use within Departments, but even here properly adhering to the plan
is vital: your colleagues in other teams will use these milestones to
align their schedules with yours, so they rely on you to be accurate.

Epics should help achieve milestones i.e. they may be blocking issues on the milestones.
A detailed description of work for a given \gls{epic} is known, it can and should be described in \gls{JIRA}.
This should be assigned to the appropriate \glspl{cycle}.



\subsection{Planning Research Work}
\label{sec:long-term-research}

In order for \RO  to reach its science goals, new algorithmic or engineering approaches must sometimes be researched.
It is appropriate to budget time for this research work in planning packages.


\subsection{Epic-Based Long Term Plans}

As long as they have not been scheduled for the current \gls{cycle}, these \glspl{epic} can be freely created and changed at any time, without any sort of approval process.

Fine grained planning of this sort can be useful for ``bottom-up''
analysis of the work to be performed and validation of the resources
needed to implement a particular planning package. Thinking through the
plan in this way can help in building up a detailed plan in a flexible,
agile way, while also ensuring that scope, cost and schedule are
carefully controlled.

\section{Short Term Planning}
\label{sec:cycle-plan}

Short term planning is carried out in blocks referred to as \glspl{cycle}, which (usually) last for six months.
Before the start of a \gls{cycle}, milestones should be confirmed by the Director's Office.

\subsection{Defining The Plan}

\subsubsection{Scoping Work}

The first essential step of developing the short term plan is to produce an outline of the program of work to be executed.
In general, this should flow directly from the long term plan (\S\ref{sec:long-term-plan}), ensuring that the expected planning packages are being worked on and milestones being hit.

While developing the \gls{cycle}, please:

\begin{itemize}
\item
  Do not add \emph{artificial} padding or buffers to make the schedule look good;
\item
  Do budget appropriate time for handling bugs and emergent issues;
\item
  Reserve time for planning the following \gls{cycle}: it will have to be defined before this \gls{cycle} is complete;
\item
  Leave time for other necessary activities, such as cross-team collaboration meetings and writing documentation.
\item
  Per the \gls{cycle} cadence (\S\ref{sec:cycle-cadence}), ensure that new development will conclude (or, at a minimum, be in a releasable state) in time for the end of \gls{cycle} release.
\end{itemize}

Obviously, ensure that the programme of work being developed is achievable by your team in the time available: ultimately, you will want to compare the number of \glspl{SP} your team is able to deliver (\S\ref{sec:effort}) with the sum of the \glspl{SP} in the \glspl{epic} you have scheduled (\S\ref{sec:planning-epics}), while also considering the skills and availability of your team.
It is better to under-commit and over-deliver than vice-versa, but, ideally, aim to estimate accurately.

\subsubsection{Defining Epics}
\label{sec:planning-epics}

The plan for a six month \gls{cycle} fundamentally consists of a set of resource loaded \glspl{epic} defined in \gls{JIRA}.
Each \gls{epic} loaded into the plan must have:

\begin{itemize}
\item
  A concrete, well defined deliverable \emph{or} be clearly described as a ``bucket'' (\S\ref{sec:bucket});
\item
  The \texttt{cycle} field set to the appropriate \gls{cycle};
\item
  The \texttt{wbs} field set to the appropriate \gls{WBS} \emph{leaf} \gls{element}.
\item
  The \texttt{Story Points} field set to a (non-zero!) estimate of the effort required to complete the \gls{epic} in terms of \glspl{SP} (see \S\ref{sec:effort}).
\end{itemize}

Be aware that:

\begin{itemize}
\item
  An \gls{epic} may only be assigned to a single \gls{cycle}.
  It is not possible to define an \gls{epic} that crosses the \gls{cycle} boundary (see \S\ref{sec:cycle-close} for the procedure when an \gls{epic} is not complete by the end of the \gls{cycle}).
\item
  An \gls{epic} may only be assigned to a single \gls{WBS} leaf \gls{element}.
  It is not possible to define \glspl{epic} that cover multiple \gls{WBS} \glspl{element}.
  See \S\ref{sec:cross-team} for information on scheduling work which requires resources from multiple \glspl{element}.
\item
  An \gls{epic} must descend from a single planning package (see \S\ref{sec:long-term-plan}).
\item
  Indeed, where possible management activities \emph{should} be scheduled as \glspl{epic} with concrete deliverables in this \gls{element} rather than being handled as \gls{LOE}.
\item
  The \gls{epic} should be at an appropriate level of granularity.
  While short \glspl{epic} (a few \glspl{SP}) may be suitable for some activities, in general \glspl{epic} will describe a few months of developer-time.
  \Glspl{epic} allocated multiple hundreds of \gls{story} points are likely too broad to be accurately estimated.
\end{itemize}


Although it is possible---indeed, encouraged---to set the \texttt{assignee} field in \gls{JIRA} to the individual who is expected to carry out the bulk of the work in an \gls{epic}, this does not provide sufficient granularity for those cases when more than one person will be contributing.

In fact, it is only required to provide a staff assignment in terms of ``resource types'' (i.e. scientists, senior scientists, developers, senior developers, etc).
In practice, to ensure your team is evenly loaded, it is usually necessary to break it down to named individuals.


\subsubsection{Scheduling Research Work}
\label{sec:research}

As discussed in \S\ref{sec:long-term-research}, research is sometimes required
to meet our objectives. However, it is not a natural fit to our usual
planning process, as it is speculative in its nature: it is often
impossible to produce a series of logical steps that will lead to the
required result. We acknowledge, therefore, that scheduling an \gls{epic} to
deliver some particular new \gls{algorithm} based on the results of research
is impossible: we cannot predict with any confidence when the
breakthrough will occur.

We therefore schedule research in \gls{timebox}ed \glspl{epic}: we allocate a certain amount of time based on the resources available, rather than on an estimate of time to completion.
However, note that these \gls{timebox}ed \glspl{epic} should still provide concrete deliverables: they are not open-ended ``buckets'' as discussed elsewhere.

\subsubsection{Bucket Epics}
\label{sec:bucket}

Some work is ``emergent'': we can predict in advance that it will be necessary, but we cannot predict exactly what form it will take.
The typical example of this is fixing bugs: we can reasonably assume that bugs will be discovered in the codebase and will need to be addressed, but we cannot predict in advance what those bugs will be.

This can be included in the schedule by defining a ``bucket'' \gls{epic} in which stories can be created when necessary during the course of a \gls{cycle}.
Make clear in the description of the \gls{epic} that this is its intended purpose: every \gls{epic} should either have a concrete deliverable or be a bucket.

Bucket \glspl{epic} have some similarities with \gls{LOE} work.
As such, we acknowledge that they are necessary, but seek to minimize the fraction of our resources assigned to them.
If more than a relatively small fraction of the work for a \gls{cycle} is assigned to bucket \glspl{epic}, please consider whether this is really necessary and appropriate.

Be aware that even bucket \glspl{epic} must be assigned to a specific \emph{leaf} \gls{element} of the \gls{WBS}.
That is, it is not in general possible to define an \gls{epic} which handles bug reports or emergent feature requests across the whole of the codebase unless a specific \gls{WBS} leaf \gls{element} is devoted to maintenance activities of this type.
Instead, it may be necessary to define a different bucket \gls{epic} for each leaf of the \gls{WBS} tree.


\subsection{Closing the Cycle}
\label{sec:cycle-close}

Assuming everything has gone to plan, by the end of a \gls{cycle} all deliverables should be verified and the corresponding \glspl{epic} should be marked as \texttt{done}.
Marking an \gls{epic} as \texttt{done} asserts that the concrete deliverable associated with the \gls{epic} has been provided.

Epics which are in progress at the end of the \gls{cycle} cannot be closed until they have been completed.
These \glspl{epic} will spill over into the subsequent \gls{cycle}.
It is \emph{not} appropriate to close an in-progress \gls{epic} with a concrete deliverable until that deliverable has been achieved: instead, a variance will be shown until the \gls{epic} can be closed.
Obviously, this will impact the labor available for other activities in the next \gls{cycle}.
(This does not apply to bucket \glspl{epic} (\S\ref{sec:bucket}), which are, by their nature, \gls{timebox}ed within the \gls{cycle}).


\section{Execution}
\label{sec:execution}

Having defined defined the plan for a \gls{cycle} following \S\ref{sec:cycle-plan}, we execute it by means of a series of month-long sprints.
In this section, we detail the procedures teams are expected to follow during the \gls{cycle}.

\subsection{Defining Stories}
\label{sec:defining-stories}

Epics have already been defined as part of the \gls{cycle} plan (see \S\ref{sec:planning-epics}).
However, the \gls{epic} is not at an appropriate level for scheduling day-to-day work.
Rather, each \gls{epic} is broken down into a series of self-contained ``stories''.
A \gls{story} describes a planned activity worth between a small fraction of a SP and several \glspl{SP} (more than about 10 is likely an indication that the \gls{story} has not been sufficiently refined).
It must be possible to schedule a \gls{story} within a single sprint, so no \gls{story} should ever be allocated more than 26 \glspl{SP}.

The process for breaking \glspl{epic} down into stories is not mandated. In
some circumstances, it may be appropriate for the technical manager to
provide a breakdown; in others, they may request input from the
developer who is actually going to be doing the work, or even hold a
brainstorming session involving the wider team. This is a management
decision.

It is not required to break all \glspl{epic} down into stories before the \gls{cycle} begins: it may be more appropriate to first schedule a few exploratory stories and use them to inform the development of the rest of the \gls{epic}.
However, do break \glspl{epic} down to describe the stories which will be worked in an upcoming sprint (\S\ref{sec:sprinting}) before the sprint starts.
When doing so, you may wish to leave some spare time to handle emergent work (discussed in \S\ref{sec:bugs}).

Note that there is no relationship enforced between the \gls{SP} total estimated for the \gls{epic} and the sum of the \glspl{SP} of its constituent stories.
It is therefore possible to over- or under-load an \gls{epic}.
This will have obvious ramifications for the schedule.

\subsection{Sprinting}
\label{sec:sprinting}

Each team organizes its work around periods of work called sprints.
A sprint comprises a defined collection of stories which will be addressed over the course of the month.
These stories are not necessarily (indeed, not generally) all drawn from the same \gls{epic}: rather, while \glspl{epic} divide the \gls{cycle} along logical grounds, sprints divide it along the time axes.

Broadly, executing a sprint falls into three stages:

\begin{enumerate}
\item
  Preparation.

  The team assigns the work that will be addressed during the sprint by choosing from the pre-defined stories (\S\ref{sec:defining-stories}).
  Each team member should be assigned a plausible amount of work, based on the per-\gls{story} \gls{SP} estimates and the likely working rate of the developer (see \S\ref{sec:effort}).

  The process by which work is assigned to team members is a local
  management decision: the orthodox approach is to call a team-wide
  meeting and discuss it, but other approaches are possible (one-to-one
  interactions between developers and technical manager, managerial
  fiat, etc).

  Do not overload developers. Take vacations and holidays into account.
  The sprint should describe a plausible amount of work for the time
  available.
\item
  Execution.

  Daily management during the sprint is a local decision. Suggested best
  practice includes holding regular ``standup'' meetings, at which
  developers discuss their current activities and try to resolve
  ``blockers'' which are preventing them from making progress.

  Stories should be executed following the instructions in the
  \href{http://developer.lsst.io/}{Developer Guide} as regards workflow,
  coding standards, review requirements, and so on. It is important to
  ensure that completed stories are marked as \texttt{done}:
  experience suggests that this can easily be forgotten as developers
  rush on to the next challenge, but it is required to enable us to
  properly track earned value as per \S\ref{sec:cycle-value}.

  When completing a \gls{story} we do not change the number of \glspl{SP} assigned to
  it: the \gls{SP} total reflects our initial estimate of the work involved,
  not the total time invested.
  However, we should \textit{also} record the true \glspl{SP} expended on the issue.
  This makes it possible to review the quality of our estimates at the end of the sprint.
  Each individual, with guidance from their \gls{T/CAM}, should use this information as they strive to improve the accuracy of their planning and estimating.

  Avoid adding more stories to a sprint in progress unless it is
  unavoidable (for example, the \gls{story} describes a critical bug that must
  be addressed before proceeding). A sprint should always stay current
  and should be up-to-date with reality; if necessary, already scheduled
  stories may be pushed out of a sprint as soon as it is obvious it is
  unrealistic to expect them to be completed.
\item
  \gls{Review}.

  At the end of the sprint, step back and consider what has been
  achieved. What worked well? What did not? How can these problems be
  avoided for next time? Was your estimate of the amount of work that
  could be finished in the sprint accurate? If not, how can it be
  improved in future? Refer to the
  \href{https://en.wikipedia.org/wiki/Burn_down_chart}{burn-down chart}
  for the sprint, and, if it diverged from the ideal, understand why.

  Again, the form the review takes is a local management decision: it
  may involve all team members, or just a few.
\end{enumerate}

We use \gls{JIRA}'s
\href{https://www.atlassian.com/software/jira/agile}{Agile} capabilities
to manage our sprints. Each technical manager is responsible for
defining and maintaining their own agile board. The board may be
configured for either
\href{https://en.wikipedia.org/wiki/Scrum_(software_development)}{Scrum}
or \href{https://en.wikipedia.org/wiki/Kanban_(development)}{Kanban}
style work as appropriate: the former is suitable for planned
development activities (e.g. \gls{Science Pipelines} development); the latter
for servicing user requests (e.g. providing developer support).

\subsection{Closing Epics}
\label{sec:epic-close}

\subsubsection{Completing the Work}
\label{sec:epic-done}

An \gls{epic} may be marked as \texttt{done} when:

\begin{enumerate}
\item
  It contains at least one completed \gls{story};
\item
  There are no more incomplete \glspl{story} defined within it;
\item
  There are no plans to add more \glspl{story};
\item
  (If applicable, i.e. it is not a bucket, as defined in \S\ref{sec:bucket}) its concrete deliverable has been achieved.
\end{enumerate}

Note that it is not permitted to close an \gls{epic} without defining at least one \gls{story} within it.
Empty \glspl{epic} can never be completed.

\subsection{Handling Bugs \& Emergent Work}
\label{sec:bugs}

\subsubsection{Receiving Bug Reports}\label{receiving-bug-reports}

Members of the project who have access to \gls{JIRA} may report bugs or make feature requests directly using \gls{JIRA}.
As discussed in \S\ref{sec:jira-maintenance}, technical managers should regularly monitor \gls{JIRA} for relevant tickets and ensure they are handled appropriately.

Our code repositories are exposed to the world in general through \href{https://github.com/lsst/}{GitHub}.
Each repository on GitHub has a bug tracker associated with it.
Members of the public may report issues or make requests on the GitHub trackers.
Per the \href{https://developer.lsst.io/processes/workflow.html}{Developer Workflow}, all new work must be associated with a \gls{JIRA} ticket number before it can be committed to the repository.
It is therefore the responsibility of technical managers to file a \gls{JIRA} ticket corresponding to the GitHub ticket, to keep them synchronized with relevant information, and to ensure that the GitHub ticket is closed when the issue is resolved in \gls{JIRA}.

The GitHub issue trackers are, in some sense, not a core part of our
workflow, but they are fundamental to community expectations of how they
can interact with the project. Ensure that issues reported on GitHub are
serviced promptly.

In some cases, the technical manager responsible for a given repository
is obvious, and they can be expected to take the lead on handling
tickets. Often, this is not the case: repositories regularly span team
boundaries. Work together to ensure that all tickets are handled.

\subsubsection{Issue Types}\label{issue-types}


\subsection{Jira Maintenance}
\label{sec:jira-maintenance}

At any time, new tickets may be added to \gls{JIRA} by team members.
Please remind your team of the best practice in this respect (\jira{RFC-147}).
It is the responsibility of technical managers to ensure that new tickets are handled appropriately, updating the schedule to include them where necessary.

It is required that the \texttt{Team} field be set to the appropriate team (\jira{RFC-145}).
This indicates which manager is responsible for \gls{seeing} that the work is completed successfully.
Available teams, and the associated managers, are listed in the \href{https://developer.lsst.io/processes/jira_agile.html}{Developer Guide}; generally speaking, they align with the the work breakdown structure described in \S\ref{sec:wbs}.
Where there is uncertainty about which team should be responsible for a particular ticket, the “System Management” team may be used to indicate that the \gls{DM} \gls{Project Manager} is responsible for assigning the work.

Please regularly monitor \gls{JIRA} for incomplete tickets and update them appropriately.
Where tickets describe bugs or other urgent emergent work which cannot be deferred, refer to \S\ref{sec:bugs}.

\subsection{Coordination Standup}
\label{sec:sup}



\section{Standard Reporting Cycle}
\label{sec:reporting-cycle}
 Cathy ?

\section{Personnel}

\subsection{Staffing Changes}
\label{sec:staffing}

In addition to onboarding procedures at your local institution, please
be aware of

\begin{itemize}
\item
  The \gls{LSST} \href{https://project.lsst.org/onboarding}{New Employee
  Onboarding} material, and
\end{itemize}

and direct new recruits to them when they join your team\footnote{As per \S\ref{sec:newcomers}, remember that newcomers should be allocated \glspl{SP} for working through this material.}.

The responsible hirere must also complete an \href{https://project.lsst.org/onboarding/form}{onboarding form} for the new recruit.
When members of staff team leave the project, the \gls{T/CAM} should fill in an \href{https://project.lsst.org/onboarding/offboarding_form}{offboarding form}.


\section{Open issues}

\begin{itemize}
\item Kanban for \gls{LOE} operations work
\item Need section on more procedural driven work on mountain and DF.
\end{itemize}
