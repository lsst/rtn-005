% DO NOT EDIT - generated by /Users/blum/GIT/lsst-texmf/bin/generateAcronyms.py from https://lsst-texmf.lsst.io/.
\newglossaryentry{} {name={}, description={}}
\newacronym{AD} {AD} {Associate Director}
\newacronym{AST} {AST} {NSF Division of Astronomical Sciences}
\newacronym{AURA} {AURA} {\gls{Association of Universities for Research in Astronomy}}
\newglossaryentry{Association of Universities for Research in Astronomy} {name={Association of Universities for Research in Astronomy}, description={ consortium of US institutions and international affiliates that operates world-class astronomical observatories, AURA is the legal entity responsible for managing what it calls independent operating Centers, including LSST, under respective cooperative agreements with the National Science Foundation. AURA assumes fiducial responsibility for the funds provided through those cooperative agreements. AURA also is the legal owner of the AURA Observatory properties in Chile}}
\newglossaryentry{Business Manager} {name={Business Manager}, description={The person responsible for all business activities of the LSST Project and the LSST Corporation; he or she serves as liaison to AURA \gls{CAS}, develops and monitors contracts, and serves as the LSST Corporation Secretary}}
\newglossaryentry{Butler} {name={Butler}, description={A middleware component for persisting and retrieving image datasets (raw or processed), calibration reference data, and catalogs}}
\newacronym{CAM} {CAM} {CAMera}
\newacronym{CAS} {CAS} {\gls{Central Administrative Services}}
\newacronym{CC} {CC} {Change Control}
\newglossaryentry{Camera} {name={Camera}, description={The LSST subsystem responsible for the 3.2-gigapixel LSST camera, which will take more than 800 panoramic images of the sky every night. SLAC leads a consortium of Department of Energy laboratories to design and build the camera sensors, optics, electronics, cryostat, filters and filter exchange mechanism, and camera control system}}
\newglossaryentry{Center} {name={Center}, description={An entity managed by AURA that is responsible for execution of a federally funded project}}
\newglossaryentry{Central Administrative Services} {name={Central Administrative Services}, description={AURA corporate division responsible for providing accounting, procurement, and business IT support services to AURA centers}}
\newglossaryentry{Change Control} {name={Change Control}, description={The systematic approach to managing all changes to the LSST system, including technical data and policy documentation. The purpose is to ensure that no unnecessary changes are made, all changes are documented, and resources are used efficiently and appropriately}}
\newglossaryentry{Construction} {name={Construction}, description={The period during which LSST observatory facilities, components, hardware, and software are built, tested, integrated, and commissioned. Construction follows design and development and precedes operations. The LSST construction phase is funded through the \gls{NSF} \gls{MREFC} account}}
\newacronym{DEI} {DEI} {Diversity, Equity, and Inclusion}
\newacronym{DF} {DF} {Data Facility}
\newacronym{DM} {DM} {\gls{Data Management}}
\newacronym{DMS} {DMS} {Data Management Subsystem}
\newacronym{DOE} {DOE} {\gls{Department of Energy}}
\newacronym{DP0} {DP0} {Data Preview 0}
\newglossaryentry{Data Management} {name={Data Management}, description={The LSST Subsystem responsible for the Data Management System (DMS), which will capture, store, catalog, and serve the LSST dataset to the scientific community and public. The DM team is responsible for the DMS architecture, applications, middleware, infrastructure, algorithms, and Observatory Network Design. DM is a distributed team working at LSST and partner institutions, with the DM Subsystem Manager located at LSST headquarters in Tucson}}
\newglossaryentry{Data Management Subsystem} {name={Data Management Subsystem}, description={The Data Management Subsystem is one of the four subsystems which constitute the LSST Construction Project. The Data Management Subsystem is responsible for developing and delivering the LSST Data Management System to the LSST Operations Project}}
\newglossaryentry{Data Management System} {name={Data Management System}, description={The computing infrastructure, middleware, and applications that process, store, and enable information extraction from the LSST dataset; the DMS will process peta-scale data volume, convert raw images into a faithful representation of the universe, and archive the results in a useful form. The infrastructure layer consists of the computing, storage, networking hardware, and system software. The middleware layer handles distributed processing, data access, user interface, and system operations services. The applications layer includes the data pipelines and the science data archives' products and services}}
\newglossaryentry{Data Release} {name={Data Release}, description={The approximately annual reprocessing of all LSST data, and the installation of the resulting data products in the LSST Data Access Centers, which marks the start of the two-year proprietary period}}
\newglossaryentry{Department of Energy} {name={Department of Energy}, description={cabinet department of the United States federal government; the DOE has assumed technical and financial responsibility for providing the LSST camera. The DOE's responsibilities are executed by a collaboration led by SLAC National Accelerator Laboratory}}
\newglossaryentry{Director} {name={Director}, description={The person responsible for the overall conduct of the project; the LSST director is charged with ensuring that both the scientific goals and management constraints on the project are met. S/he is the principal public spokesperson for the project in all matters and represents the project to the scientific community, AURA, the member institutions of LSSTC, and the funding agencies}}
\newglossaryentry{DocuShare} {name={DocuShare}, description={The trade name for the enterprise management software used by LSST to archive and manage documents}}
\newglossaryentry{Document} {name={Document}, description={Any object (in any application supported by DocuShare or design archives such as PDMWorks or GIT) that supports project management or records milestones and deliverables of the LSST Project}}
\newacronym{EPO} {EPO} {Education and Public Outreach}
\newacronym{EVM} {EVM} {Earned Value Management}
\newacronym{EVMS} {EVMS} {Earned Value Management System}
\newglossaryentry{Earned Value Management} {name={Earned Value Management}, description={A project management technique for objectively measuring project performance and progress in terms of budget and schedule}}
\newglossaryentry{Earned Value Management System} {name={Earned Value Management System}, description={A set of tools, techniques and procedures which are used to implement a \gls{EVM} approach to project management}}
\newglossaryentry{Education and Public Outreach} {name={Education and Public Outreach}, description={The LSST subsystem responsible for the cyberinfrastructure, user interfaces, and outreach programs necessary to connect educators, planetaria, citizen scientists, amateur astronomers, and the general public to the transformative LSST dataset}}
\newacronym{FTE} {FTE} {Full-Time Equivalent}
\newacronym{FWP} {FWP} {Field Work Proposals}
\newacronym{FY} {FY} {Financial Year}
\newglossaryentry{Full-Time Equivalent} {name={Full-Time Equivalent}, description={A unit equivalent to one person working full time for one year with normal holidays, vacations, and sick time. No paid overtime is assumed}}
\newacronym{HEP} {HEP} { High Energy Physics}
\newglossaryentry{Handle} {name={Handle}, description={The unique identifier assigned to a document uploaded to DocuShare}}
\newacronym{IDF} {IDF} {Interim Data Facility}
\newacronym{IT} {IT} {Information Technology}
\newglossaryentry{JIRA} {name={JIRA}, description={issue tracking product (not an acronym but a truncation of Gojira the Japanese name for Godzilla)}}
\newacronym{L1} {L1} {Lens 1}
\newacronym{L2} {L2} {Lens 2}
\newacronym{L3} {L3} {Lens 3}
\newacronym{LDM} {LDM} {LSST Data Management (Document Handle)}
\newacronym{LOE} {LOE} {Level of Effort}
\newacronym{LSST} {LSST} {Legacy Survey of Space and Time (formerly Large Synoptic Survey Telescope)}
\newglossaryentry{LSST Corporation} {name={LSST Corporation}, description={An Arizona 501(c)3 not-for-profit corporation formed in 2003 for the purpose of designing, constructing, and operating the LSST System. During design and development, the Corporation stewarded private funding used for such essential contributions as early site preparation, mirror construction, and early data management system development. During construction, LSSTC will secure private operations funding from international affiliates and play a key role in preparing the scientific community to use the LSST dataset}}
\newglossaryentry{LSST Project Office} {name={LSST Project Office}, description={Official name of the stand-alone AURA operating center responsible for execution of the LSST construction project under the NSF \gls{MREFC} account}}
\newacronym{LSSTC} {LSSTC} {\gls{LSST} Corporation}
\newacronym{LSSTPO} {LSSTPO} {\gls{LSST} Project Office}
\newacronym{MPO} {MPO} {Memorandum Purchase Order}
\newacronym{MREFC} {MREFC} {\gls{Major Research Equipment and Facility Construction}}
\newglossaryentry{Major Research Equipment and Facility Construction} {name={Major Research Equipment and Facility Construction}, description={the NSF account through which large facilities construction projects such as LSST are funded}}
\newacronym{NSF} {NSF} {\gls{National Science Foundation}}
\newglossaryentry{National Science Foundation} {name={National Science Foundation}, description={primary federal agency supporting research in all fields of fundamental science and engineering; NSF selects and funds projects through competitive, merit-based review}}
\newglossaryentry{Operations} {name={Operations}, description={The 10-year period following construction and commissioning during which the LSST Observatory conducts its survey}}
\newacronym{PCW} {PCW} {Project Community Workshop}
\newacronym{POP} {POP} {Project Operating Plan}
\newacronym{PSF} {PSF} {Point Spread Function}
\newglossaryentry{Project Manager} {name={Project Manager}, description={The person responsible for exercising leadership and oversight over the entire LSST project; he or she controls schedule, budget, and all contingency funds}}
\newacronym{QA} {QA} {Quality Assurance}
\newacronym{QC} {QC} {Quality Control}
\newglossaryentry{Quality Assurance} {name={Quality Assurance}, description={All activities, deliverables, services, documents, procedures or artifacts which are designed to ensure the quality of DM deliverables. This may include \gls{QC} systems, in so far as they are covered in the charge described in LDM-622. Note that contrasts with the LDM-522 definition of “QA” as “Quality Analysis”, a manual process which occurs only during commissioning and operations. See also: Quality Control}}
\newglossaryentry{Quality Control} {name={Quality Control}, description={Services and processes which are aimed at measuring and monitoring a system to verify and characterize its performance (as in LDM-522). Quality Control systems run autonomously, only notifying people when an anomaly has been detected. See also Quality Assurance}}
\newacronym{RDO} {RDO} {Rubin Directors Office}
\newacronym{RDP} {RDP} {Rubin Data Production}
\newacronym{REO} {REO} {Rubin Education and Outreach}
\newacronym{RFC} {RFC} {Request For Comment}
\newacronym{ROO} {ROO} {Rubin Observatory Operations}
\newacronym{RSS} {RSS} {square root of the sum of the squares}
\newacronym{RTN} {RTN} {Rubin Technical Note}
\newglossaryentry{Release} {name={Release}, description={Publication of a new version of a document, software, or data product. Depending on context, releases may require approval from Project- or DM-level change control boards, and then form part of the formal project baseline}}
\newglossaryentry{Review} {name={Review}, description={Programmatic and/or technical audits of a given component of the project, where a preferably independent committee advises further project decisions, based on the current status and their evaluation of it. The reviews assess technical performance and maturity, as well as the compliance of the design and end product with the stated requirements and interfaces}}
\newglossaryentry{Risk} {name={Risk}, description={The degree of exposure to an event that might happen to the detriment of a program, project, or other activity. It is described by a combination of the probability that the risk event will occur and the consequence of the extent of loss from the occurrence, or impact. Risk is an inherent part of all activities, whether the activity is simple and small, or large and complex}}
\newacronym{SLAC} {SLAC} {SLAC National Accelerator Laboratory (formerly Stanford Linear Accelerator Center; SLAC is now no longer an acronym)}
\newacronym{SP} {SP} {Survey Performance}
\newglossaryentry{Safety} {name={Safety}, description={The control of accidental loss}}
\newglossaryentry{Science Pipelines} {name={Science Pipelines}, description={The library of software components and the algorithms and processing pipelines assembled from them that are being developed by DM to generate science-ready data products from LSST images. The Pipelines may be executed at scale as part of LSST Prompt or Data Release processing, or pieces of them may be used in a standalone mode or executed through the LSST Science Platform. The Science Pipelines are one component of the LSST Software Stack}}
\newglossaryentry{Science Platform} {name={Science Platform}, description={A set of integrated web applications and services deployed at the LSST Data Access Centers (DACs) through which the scientific community will access, visualize, and perform next-to-the-data analysis of the LSST data products}}
\newglossaryentry{Software Stack} {name={Software Stack}, description={Often referred to as the LSST Stack, or just The Stack, it is the collection of software written by the LSST Data Management Team to process, generate, and serve LSST images, transient alerts, and catalogs. The Stack includes the LSST Science Pipelines, as well as packages upon which the DM software depends. It is open source and publicly available}}
\newglossaryentry{Subsystem} {name={Subsystem}, description={A set of elements comprising a system within the larger LSST system that is responsible for a key technical deliverable of the project}}
\newglossaryentry{Subsystem Manager} {name={Subsystem Manager}, description={responsible manager for an LSST subsystem; he or she exercises authority, within prescribed limits and under scrutiny of the Project Manager, over the relevant subsystem's cost, schedule, and work plans}}
\newglossaryentry{Summit} {name={Summit}, description={The site on the Cerro Pach\'{o}n, Chile mountaintop where the LSST observatory, support facilities, and infrastructure will be built}}
\newglossaryentry{Systems Engineering} {name={Systems Engineering}, description={an interdisciplinary field of engineering that focuses on how to design and manage complex engineering systems over their life cycles. Issues such as requirements engineering, reliability, logistics, coordination of different teams, testing and evaluation, maintainability and many other disciplines necessary for successful system development, design, implementation, and ultimate decommission become more difficult when dealing with large or complex projects. Systems engineering deals with work-processes, optimization methods, and risk management tools in such projects. It overlaps technical and human-centered disciplines such as industrial engineering, control engineering, software engineering, organizational studies, and project management. Systems engineering ensures that all likely aspects of a project or system are considered, and integrated into a whole}}
\newacronym{T/CAM} {T/CAM} {Technical/Control (or Cost) Account Manager}
\newacronym{UK} {UK} {United Kingdom}
\newacronym{US} {US} {United States}
\newacronym{VRO} {VRO} {(not to be used)Vera C. Rubin Observatory}
\newglossaryentry{Validation} {name={Validation}, description={A process of confirming that the delivered system will provide its desired functionality; overall, a validation process includes the evaluation, integration, and test activities carried out at the system level to ensure that the final developed system satisfies the intent and performance of that system in operations}}
\newglossaryentry{Verification} {name={Verification}, description={The process of evaluating the design, including hardware and software - to ensure the requirements have been met;  verification (of requirements) is performed by test, analysis, inspection, and/or demonstration}}
\newacronym{WBS} {WBS} {\gls{Work Breakdown Structure}}
\newglossaryentry{Work Breakdown Structure} {name={Work Breakdown Structure}, description={a tool that defines and organizes the LSST project's total work scope through the enumeration and grouping of the project's discrete work elements}}
\newglossaryentry{airmass} {name={airmass}, description={The pathlength of light from an astrophysical source through the Earth's atmosphere. It is given approximately by sec z, where z is the angular distance from the zenith (the point directly overhead, where airmass = 1.0) to the source}}
\newglossaryentry{algorithm} {name={algorithm}, description={A computational implementation of a calculation or some method of processing}}
\newglossaryentry{astronomical object} {name={astronomical object}, description={A star, galaxy, asteroid, or other physical object of astronomical interest. Beware: in non-LSST usage, these are often known as sources}}
\newglossaryentry{calibration} {name={calibration}, description={The process of translating signals produced by a measuring instrument such as a telescope and camera into physical units such as flux, which are used for scientific analysis. Calibration removes most of the contributions to the signal from environmental and instrumental factors, such that only the astronomical component remains}}
\newglossaryentry{camera} {name={camera}, description={An imaging device mounted at a telescope focal plane, composed of optics, a shutter, a set of filters, and one or more sensors arranged in a focal plane array}}
\newglossaryentry{cycle} {name={cycle}, description={The time period over which detailed, short-term plans are defined and executed. Normally, cycles run for six months, and culminate in a new release of the LSST Software Stack, however this need not always be the case}}
\newglossaryentry{element} {name={element}, description={A node in the hierarchical project \gls{WBS}}}
\newglossaryentry{epic} {name={epic}, description={A self contained work with a concrete deliverable which my be scheduled to take place with a single \gls{cycle} and \gls{WBS} \gls{element}}}
\newglossaryentry{flux} {name={flux}, description={Shorthand for radiative flux, it is a measure of the transport of radiant energy per unit area per unit time. In astronomy this is usually expressed in cgs units: erg/cm2/s}}
\newglossaryentry{middleware} {name={middleware}, description={Software that acts as a bridge between other systems or software usually a database or network. Specifically in the Data Management System this refers to Butler for data access and Workflow management for distributed processing.}}
\newglossaryentry{monitoring} {name={monitoring}, description={In DM QA, this refers to the process of collecting, storing, aggregating and visualizing metrics}}
\newglossaryentry{passband} {name={passband}, description={The window of wavelength or the energy range admitted by an optical system; specifically the transmission as a function of wavelength or energy. Typically the passband is limited by a filter. The width of the passband may be characterized in a variety of ways, including the width of the half-power points of the transmission curve, or by the equivalent width of a filter with 100\% transmission within the passband, and zero elsewhere}}
\newglossaryentry{seeing} {name={seeing}, description={An astronomical term for characterizing the stability of the atmosphere, as measured by the width of the point-spread function on images. The PSF width is also affected by a number of other factors, including the airmass, passband, and the telescope and camera optics}}
\newglossaryentry{story} {name={story}, description={A \gls{JIRA} issue type describing a scheduled, self-contained task worked as part of an \gls{epic}.  Typically, stories are appropriate for work worth between a fraction of a \gls{SP} and 10 \glspl{SP}; beyond that, the work is insufficiently fine-grained to schedule as a story.  While fractional \glspl{SP} are fine, all stories involve work, so the \glspl{SP} total of an in progress or completed story should not be 0}}
\newglossaryentry{timebox} {name={timebox}, description={A limited time period assigned to a piece of work or other activity.  Useful in scheduling work which is not otherwise easily limited in scope, for example research projects or servicing user requests}}
\newglossaryentry{transient} {name={transient}, description={A transient source is one that has been detected on a difference image, but has not been associated with either an astronomical object or a solar system body}}
\newglossaryentry{NOIRLab} {name={NOIRLab}, description={NSF's National Optical Infrared Astronomy Research Laboratory}}
