% DO NOT EDIT - generated by /Users/womullan/LSSTgit/lsst-texmf/bin/generateAcronyms.py from https://lsst-texmf.lsst.io/.
\newacronym{AURA} {AURA} {\gls{Association of Universities for Research in Astronomy}}
\newglossaryentry{Archive} {name={Archive}, description={The repository for documents required by the NSF to be kept. These include documents related to design and development, construction, integration, test, and operations of the LSST observatory system. The archive is maintained using the enterprise content management system DocuShare, which is accessible through a link on the project website www.project.lsst.org}}
\newglossaryentry{Archive Center} {name={Archive Center}, description={Part of the LSST Data Management System, the LSST archive center is a data center at NCSA that hosts the LSST Archive, which includes released science data and metadata, observatory and engineering data, and supporting software such as the LSST Software Stack}}
\newglossaryentry{Association of Universities for Research in Astronomy} {name={Association of Universities for Research in Astronomy}, description={ consortium of US institutions and international affiliates that operates world-class astronomical observatories, AURA is the legal entity responsible for managing what it calls independent operating Centers, including LSST, under respective cooperative agreements with the National Science Foundation. AURA assumes fiducial responsibility for the funds provided through those cooperative agreements. AURA also is the legal owner of the AURA Observatory properties in Chile}}
\newacronym{B} {B} {Byte (8 bit)}
\newglossaryentry{Butler} {name={Butler}, description={A middleware component for persisting and retrieving image datasets (raw or processed), calibration reference data, and catalogs}}
\newacronym{CA} {CA} {Control (or Cost) Account}
\newacronym{CAM} {CAM} {CAMera}
\newglossaryentry{Center} {name={Center}, description={An entity managed by AURA that is responsible for execution of a federally funded project}}
\newglossaryentry{Construction} {name={Construction}, description={The period during which LSST observatory facilities, components, hardware, and software are built, tested, integrated, and commissioned. Construction follows design and development and precedes operations. The LSST construction phase is funded through the \gls{NSF} \gls{MREFC} account}}
\newacronym{DESC} {DESC} {Dark Energy Science Collaboration}
\newglossaryentry{DIAObject} {name={DIAObject}, description={A DIAObject is the association of DIASources, by coordinate, that have been detected with signal-to-noise ratio greater than 5 in at least one difference image. It is distinguished from a regular Object in that its brightness varies in time, and from a SSObject in that it is stationary (non-moving)}}
\newglossaryentry{DIASource} {name={DIASource}, description={A DIASource is a detection with signal-to-noise ratio greater than 5 in a difference image}}
\newacronym{DM} {DM} {\gls{Data Management}}
\newacronym{DMS} {DMS} {Data Management Subsystem}
\newacronym{DR} {DR} {Data Release}
\newacronym{DRP} {DRP} {Data Release Production}
\newglossaryentry{Data Management} {name={Data Management}, description={The LSST Subsystem responsible for the Data Management System (DMS), which will capture, store, catalog, and serve the LSST dataset to the scientific community and public. The DM team is responsible for the DMS architecture, applications, middleware, infrastructure, algorithms, and Observatory Network Design. DM is a distributed team working at LSST and partner institutions, with the DM Subsystem Manager located at LSST headquarters in Tucson}}
\newglossaryentry{Data Management Subsystem} {name={Data Management Subsystem}, description={The Data Management Subsystem is one of the four subsystems which constitute the LSST Construction Project. The Data Management Subsystem is responsible for developing and delivering the LSST Data Management System to the LSST Operations Project}}
\newglossaryentry{Data Management System} {name={Data Management System}, description={The computing infrastructure, middleware, and applications that process, store, and enable information extraction from the LSST dataset; the DMS will process peta-scale data volume, convert raw images into a faithful representation of the universe, and archive the results in a useful form. The infrastructure layer consists of the computing, storage, networking hardware, and system software. The middleware layer handles distributed processing, data access, user interface, and system operations services. The applications layer includes the data pipelines and the science data archives' products and services}}
\newglossaryentry{Data Release} {name={Data Release}, description={The approximately annual reprocessing of all LSST data, and the installation of the resulting data products in the LSST Data Access Centers, which marks the start of the two-year proprietary period}}
\newglossaryentry{Data Release Production} {name={Data Release Production}, description={An episode of (re)processing all of the accumulated LSST images, during which all output DR data products are generated. These episodes are planned to occur annually during the LSST survey, and the processing will be executed at the Archive Center. This includes Difference Imaging Analysis, generating deep Coadd Images, Source detection and association, creating Object and Solar System Object catalogs, and related metadata}}
\newglossaryentry{Deputy Director} {name={Deputy Director}, description={The person who supports the Director in the execution of the overall LSST project and assumes his or her duties and authority during any short term or extended absence, planned or unplanned}}
\newglossaryentry{Director} {name={Director}, description={The person responsible for the overall conduct of the project; the LSST director is charged with ensuring that both the scientific goals and management constraints on the project are met. S/he is the principal public spokesperson for the project in all matters and represents the project to the scientific community, AURA, the member institutions of LSSTC, and the funding agencies}}
\newglossaryentry{DocuShare} {name={DocuShare}, description={The trade name for the enterprise management software used by LSST to archive and manage documents}}
\newglossaryentry{Document} {name={Document}, description={Any object (in any application supported by DocuShare or design archives such as PDMWorks or GIT) that supports project management or records milestones and deliverables of the LSST Project}}
\newacronym{EVM} {EVM} {Earned Value Management}
\newacronym{EVMS} {EVMS} {Earned Value Management System}
\newglossaryentry{Earned Value Management} {name={Earned Value Management}, description={A project management technique for objectively measuring project performance and progress in terms of budget and schedule}}
\newglossaryentry{Earned Value Management System} {name={Earned Value Management System}, description={A set of tools, techniques and procedures which are used to implement a \gls{EVM} approach to project management}}
\newacronym{FITS} {FITS} {\gls{Flexible Image Transport System}}
\newglossaryentry{Flexible Image Transport System} {name={Flexible Image Transport System}, description={an international standard in astronomy for storing images, tables, and metadata in disk files. See the IAU FITS Standard for details}}
\newglossaryentry{Handle} {name={Handle}, description={The unique identifier assigned to a document uploaded to DocuShare}}
\newacronym{IAU} {IAU} {International Astronomical Union}
\newglossaryentry{JIRA} {name={JIRA}, description={issue tracking product (not an acronym but a truncation of Gojira the Japanese name for Godzilla)}}
\newacronym{LDO} {LDO} {LSST Document Operations (Document Handle)}
\newacronym{LOE} {LOE} {Level of Effort}
\newacronym{LSST} {LSST} {Legacy Survey of Space and Time (formerly Large Synoptic Survey Telescope)}
\newacronym{LSSTC} {LSSTC} {\gls{LSST} Corporation}
\newacronym{MREFC} {MREFC} {\gls{Major Research Equipment and Facility Construction}}
\newglossaryentry{Major Research Equipment and Facility Construction} {name={Major Research Equipment and Facility Construction}, description={the NSF account through which large facilities construction projects such as LSST are funded}}
\newacronym{NCSA} {NCSA} {National Center for Supercomputing Applications}
\newacronym{NSF} {NSF} {\gls{National Science Foundation}}
\newglossaryentry{National Science Foundation} {name={National Science Foundation}, description={primary federal agency supporting research in all fields of fundamental science and engineering; NSF selects and funds projects through competitive, merit-based review}}
\newacronym{OBS} {OBS} {Organisation Breakdown Structure}
\newglossaryentry{Object} {name={Object}, description={In LSST nomenclature this refers to an astronomical object, such as a star, galaxy, or other physical entity. E.g., comets, asteroids are also Objects but typically called a Moving Object or a Solar System Object (SSObject). One of the DRP data products is a table of Objects detected by LSST which can be static, or change brightness or position with time}}
\newglossaryentry{Operations} {name={Operations}, description={The 10-year period following construction and commissioning during which the LSST Observatory conducts its survey}}
\newacronym{PSF} {PSF} {Point Spread Function}
\newglossaryentry{Primavera} {name={Primavera}, description={The trade name for the project management software suite used by LSST to maintain its program plan and schedule}}
\newglossaryentry{Project Manager} {name={Project Manager}, description={The person responsible for exercising leadership and oversight over the entire LSST project; he or she controls schedule, budget, and all contingency funds}}
\newacronym{RFC} {RFC} {Request For Comment}
\newacronym{RTN} {RTN} {Rubin Technical Note}
\newglossaryentry{Review} {name={Review}, description={Programmatic and/or technical audits of a given component of the project, where a preferably independent committee advises further project decisions, based on the current status and their evaluation of it. The reviews assess technical performance and maturity, as well as the compliance of the design and end product with the stated requirements and interfaces}}
\newacronym{SP} {SP} {Survey Performance}
\newglossaryentry{Science Collaboration} {name={Science Collaboration}, description={An autonomous body of scientists interested in a particular area of science enabled by the LSST dataset, which through precursor studies, simulations, and algorithm development lays the groundwork for the large-scale science projects the LSST will enable.  In addition to preparing their members to take full advantage of LSST early in its operations phase, the science collaborations have helped to define the system's science requirements, refine and promote the science case, and quality check design and development work}}
\newglossaryentry{Science Pipelines} {name={Science Pipelines}, description={The library of software components and the algorithms and processing pipelines assembled from them that are being developed by DM to generate science-ready data products from LSST images. The Pipelines may be executed at scale as part of LSST Prompt or Data Release processing, or pieces of them may be used in a standalone mode or executed through the LSST Science Platform. The Science Pipelines are one component of the LSST Software Stack}}
\newglossaryentry{Science Platform} {name={Science Platform}, description={A set of integrated web applications and services deployed at the LSST Data Access Centers (DACs) through which the scientific community will access, visualize, and perform next-to-the-data analysis of the LSST data products}}
\newglossaryentry{Software Stack} {name={Software Stack}, description={Often referred to as the LSST Stack, or just The Stack, it is the collection of software written by the LSST Data Management Team to process, generate, and serve LSST images, transient alerts, and catalogs. The Stack includes the LSST Science Pipelines, as well as packages upon which the DM software depends. It is open source and publicly available}}
\newglossaryentry{Solar System Object} {name={Solar System Object}, description={A solar system object is an astrophysical object that is identified as part of the Solar System: planets and their satellites, asteroids, comets, etc. This class of object had historically been referred to within the LSST Project as Moving Objects}}
\newglossaryentry{Source} {name={Source}, description={A single detection of an astrophysical object in an image, the characteristics for which are stored in the Source Catalog of the DRP database. The association of Sources that are non-moving lead to Objects; the association of moving Sources leads to Solar System Objects. (Note that in non-LSST usage "source" is often used for what LSST calls an Object.)}}
\newglossaryentry{Subsystem} {name={Subsystem}, description={A set of elements comprising a system within the larger LSST system that is responsible for a key technical deliverable of the project}}
\newglossaryentry{Subsystem Manager} {name={Subsystem Manager}, description={responsible manager for an LSST subsystem; he or she exercises authority, within prescribed limits and under scrutiny of the Project Manager, over the relevant subsystem's cost, schedule, and work plans}}
\newacronym{T/CAM} {T/CAM} {Technical/Control (or Cost) Account Manager}
\newacronym{US} {US} {United States}
\newacronym{VRO} {VRO} {(not to be used)Vera C. Rubin Observatory}
\newacronym{WBS} {WBS} {\gls{Work Breakdown Structure}}
\newglossaryentry{Work Breakdown Structure} {name={Work Breakdown Structure}, description={a tool that defines and organizes the LSST project's total work scope through the enumeration and grouping of the project's discrete work elements}}
\newglossaryentry{airmass} {name={airmass}, description={The pathlength of light from an astrophysical source through the Earth's atmosphere. It is given approximately by sec z, where z is the angular distance from the zenith (the point directly overhead, where airmass = 1.0) to the source}}
\newglossaryentry{algorithm} {name={algorithm}, description={A computational implementation of a calculation or some method of processing}}
\newglossaryentry{astronomical object} {name={astronomical object}, description={A star, galaxy, asteroid, or other physical object of astronomical interest. Beware: in non-LSST usage, these are often known as sources}}
\newglossaryentry{calibration} {name={calibration}, description={The process of translating signals produced by a measuring instrument such as a telescope and camera into physical units such as flux, which are used for scientific analysis. Calibration removes most of the contributions to the signal from environmental and instrumental factors, such that only the astronomical component remains}}
\newglossaryentry{camera} {name={camera}, description={An imaging device mounted at a telescope focal plane, composed of optics, a shutter, a set of filters, and one or more sensors arranged in a focal plane array}}
\newglossaryentry{cycle} {name={cycle}, description={The time period over which detailed, short-term plans are defined and executed. Normally, cycles run for six months, and culminate in a new release of the LSST Software Stack, however this need not always be the case}}
\newglossaryentry{element} {name={element}, description={A node in the hierarchical project \gls{WBS}}}
\newglossaryentry{epic} {name={epic}, description={A self contained work with a concrete deliverable which my be scheduled to take place with a single \gls{cycle} and \gls{WBS} \gls{element}}}
\newglossaryentry{flux} {name={flux}, description={Shorthand for radiative flux, it is a measure of the transport of radiant energy per unit area per unit time. In astronomy this is usually expressed in cgs units: erg/cm2/s}}
\newglossaryentry{metadata} {name={metadata}, description={General term for data about data, e.g., attributes of astronomical objects (e.g. images, sources, astroObjects, etc.) that are characteristics of the objects themselves, and facilitate the organization, preservation, and query of data sets. (E.g., a FITS header contains metadata)}}
\newglossaryentry{middleware} {name={middleware}, description={Software that acts as a bridge between other systems or software usually a database or network. Specifically in the Data Management System this refers to Butler for data access and Workflow management for distributed processing.}}
\newglossaryentry{passband} {name={passband}, description={The window of wavelength or the energy range admitted by an optical system; specifically the transmission as a function of wavelength or energy. Typically the passband is limited by a filter. The width of the passband may be characterized in a variety of ways, including the width of the half-power points of the transmission curve, or by the equivalent width of a filter with 100\% transmission within the passband, and zero elsewhere}}
\newglossaryentry{seeing} {name={seeing}, description={An astronomical term for characterizing the stability of the atmosphere, as measured by the width of the point-spread function on images. The PSF width is also affected by a number of other factors, including the airmass, passband, and the telescope and camera optics}}
\newglossaryentry{shape} {name={shape}, description={In reference to a Source or Object, the shape is a functional characterization of its spatial intensity distribution, and the integral of the shape is the flux. Shape characterizations are a data product in the DIASource, DIAObject, Source, and Object catalogs}}
\newglossaryentry{story} {name={story}, description={A \gls{JIRA} issue type describing a scheduled, self-contained task worked as part of an \gls{epic}.  Typically, stories are appropriate for work worth between a fraction of a \gls{SP} and 10 \glspl{SP}; beyond that, the work is insufficiently fine-grained to schedule as a story.  While fractional \glspl{SP} are fine, all stories involve work, so the \glspl{SP} total of an in progress or completed story should not be 0}}
\newglossaryentry{timebox} {name={timebox}, description={A limited time period assigned to a piece of work or other activity.  Useful in scheduling work which is not otherwise easily limited in scope, for example research projects or servicing user requests}}
\newglossaryentry{transient} {name={transient}, description={A transient source is one that has been detected on a difference image, but has not been associated with either an astronomical object or a solar system body}}
