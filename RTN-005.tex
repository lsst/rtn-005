\documentclass[DM,authoryear,toc]{lsstdoc}
% lsstdoc documentation: https://lsst-texmf.lsst.io/lsstdoc.html

\input{meta}

% Package imports go here.

% Local commands go here.

%If you want glossaries
% DO NOT EDIT - generated by /Users/womullan/LSSTgit/lsst-texmf/bin/generateAcronyms.py from https://lsst-texmf.lsst.io/.
\newacronym{AD} {AD} {Associate \gls{Director}}
\newacronym{AST} {AST} {NSF Division of Astronomical Sciences}
\newacronym{AURA} {AURA} {\gls{Association of Universities for Research in Astronomy}}
\newglossaryentry{Association of Universities for Research in Astronomy} {name={Association of Universities for Research in Astronomy}, description={ consortium of US institutions and international affiliates that operates world-class astronomical observatories, AURA is the legal entity responsible for managing what it calls independent operating Centers, including LSST, under respective cooperative agreements with the National Science Foundation. AURA assumes fiducial responsibility for the funds provided through those cooperative agreements. AURA also is the legal owner of the AURA Observatory properties in Chile}}
\newglossaryentry{Business Manager} {name={Business Manager}, description={The person responsible for all business activities of the LSST Project and the LSST Corporation; he or she serves as liaison to AURA CAS, develops and monitors contracts, and serves as the LSST Corporation Secretary}}
\newglossaryentry{Butler} {name={Butler}, description={A middleware component for persisting and retrieving image datasets (raw or processed), calibration reference data, and catalogs}}
\newacronym{CAM} {CAM} {CAMera}
\newacronym{CAS} {CAS} {\gls{Central Administrative Services}}
\newacronym{CC} {CC} {\gls{Change Control}}
\newglossaryentry{Camera} {name={Camera}, description={The LSST subsystem responsible for the 3.2-gigapixel LSST camera, which will take more than 800 panoramic images of the sky every night. SLAC leads a consortium of Department of Energy laboratories to design and build the camera sensors, optics, electronics, cryostat, filters and filter exchange mechanism, and camera control system}}
\newglossaryentry{Central Administrative Services} {name={Central Administrative Services}, description={AURA corporate division responsible for providing accounting, procurement, and business IT support services to AURA centers}}
\newglossaryentry{Change Control} {name={Change Control}, description={The systematic approach to managing all changes to the LSST system, including technical data and policy documentation. The purpose is to ensure that no unnecessary changes are made, all changes are documented, and resources are used efficiently and appropriately}}
\newglossaryentry{Construction} {name={Construction}, description={The period during which LSST observatory facilities, components, hardware, and software are built, tested, integrated, and commissioned. Construction follows design and development and precedes operations. The LSST construction phase is funded through the NSF MREFC account}}
\newacronym{DEI} {DEI} {Diversity, Equity, and Inclusion}
\newacronym{DF} {DF} {Data Facility}
\newacronym{DM} {DM} {\gls{Data Management}}
\newacronym{DMS} {DMS} {\gls{Data Management Subsystem}}
\newacronym{DOE} {DOE} {\gls{Department of Energy}}
\newacronym{DP0} {DP0} {Data Preview 0}
\newacronym{DPRPT} {DPRPT} {Data Preview and \gls{Release} Planning Tool}
\newglossaryentry{Data Management} {name={Data Management}, description={The LSST Subsystem responsible for the Data Management System (DMS), which will capture, store, catalog, and serve the LSST dataset to the scientific community and public. The DM team is responsible for the DMS architecture, applications, middleware, infrastructure, algorithms, and Observatory Network Design. DM is a distributed team working at LSST and partner institutions, with the DM Subsystem Manager located at LSST headquarters in Tucson}}
\newglossaryentry{Data Management Subsystem} {name={Data Management Subsystem}, description={The Data Management Subsystem is one of the four subsystems which constitute the LSST Construction Project. The Data Management Subsystem is responsible for developing and delivering the LSST Data Management System to the LSST Operations Project}}
\newglossaryentry{Data Management System} {name={Data Management System}, description={The computing infrastructure, middleware, and applications that process, store, and enable information extraction from the LSST dataset; the DMS will process peta-scale data volume, convert raw images into a faithful representation of the universe, and archive the results in a useful form. The infrastructure layer consists of the computing, storage, networking hardware, and system software. The middleware layer handles distributed processing, data access, user interface, and system operations services. The applications layer includes the data pipelines and the science data archives' products and services}}
\newglossaryentry{Data Release} {name={Data Release}, description={The approximately annual reprocessing of all LSST data, and the installation of the resulting data products in the LSST Data Access Centers, which marks the start of the two-year proprietary period}}
\newglossaryentry{Department of Energy} {name={Department of Energy}, description={cabinet department of the United States federal government; the DOE has assumed technical and financial responsibility for providing the LSST camera. The DOE's responsibilities are executed by a collaboration led by SLAC National Accelerator Laboratory}}
\newglossaryentry{Director} {name={Director}, description={The person responsible for the overall conduct of the project; the LSST director is charged with ensuring that both the scientific goals and management constraints on the project are met. S/he is the principal public spokesperson for the project in all matters and represents the project to the scientific community, AURA, the member institutions of LSSTC, and the funding agencies}}
\newglossaryentry{DocuShare} {name={DocuShare}, description={The trade name for the enterprise management software used by LSST to archive and manage documents}}
\newglossaryentry{Document} {name={Document}, description={Any object (in any application supported by DocuShare or design archives such as PDMWorks or GIT) that supports project management or records milestones and deliverables of the LSST Project}}
\newacronym{EPO} {EPO} {\gls{Education and Public Outreach}}
\newacronym{ES} {ES} {Early Science}
\newacronym{EVM} {EVM} {\gls{Earned Value Management}}
\newacronym{EVMS} {EVMS} {\gls{Earned Value Management System}}
\newglossaryentry{Earned Value Management} {name={Earned Value Management}, description={A project management technique for objectively measuring project performance and progress in terms of budget and schedule}}
\newglossaryentry{Earned Value Management System} {name={Earned Value Management System}, description={A set of tools, techniques and procedures which are used to implement a EVM approach to project management}}
\newglossaryentry{Education and Public Outreach} {name={Education and Public Outreach}, description={The LSST subsystem responsible for the cyberinfrastructure, user interfaces, and outreach programs necessary to connect educators, planetaria, citizen scientists, amateur astronomers, and the general public to the transformative LSST dataset}}
\newacronym{FTE} {FTE} {\gls{Full-Time Equivalent}}
\newacronym{FWP} {FWP} {Field Work Proposals}
\newacronym{FY} {FY} {Financial Year}
\newacronym{FY23} {FY23} {Financial Year 23}
\newacronym{FY24} {FY24} {Financial Year 24}
\newglossaryentry{Full-Time Equivalent} {name={Full-Time Equivalent}, description={A unit equivalent to one person working full time for one year with normal holidays, vacations, and sick time. No paid overtime is assumed}}
\newacronym{HEP} {HEP} { High Energy Physics}
\newglossaryentry{Handle} {name={Handle}, description={The unique identifier assigned to a document uploaded to DocuShare}}
\newacronym{IDF} {IDF} {Interim Data Facility}
\newacronym{IT} {IT} {Information Technology}
\newglossaryentry{JIRA} {name={JIRA}, description={issue tracking product (not an acronym but a truncation of Gojira the Japanese name for Godzilla)}}
\newacronym{JOR} {JOR} {Joint Operations status \gls{Review}}
\newacronym{L1} {L1} {Lens 1}
\newacronym{L2} {L2} {Lens 2}
\newacronym{L3} {L3} {Lens 3}
\newacronym{LDM} {LDM} {LSST Data Management (Document \gls{Handle})}
\newacronym{LOE} {LOE} {Level of Effort}
\newacronym{LSST} {LSST} {Legacy Survey of Space and Time (formerly Large Synoptic Survey Telescope)}
\newglossaryentry{LSST Corporation} {name={LSST Corporation}, description={An Arizona 501(c)3 not-for-profit corporation formed in 2003 for the purpose of designing, constructing, and operating the LSST System. During design and development, the Corporation stewarded private funding used for such essential contributions as early site preparation, mirror construction, and early data management system development. During construction, LSSTC will secure private operations funding from international affiliates and play a key role in preparing the scientific community to use the LSST dataset}}
\newglossaryentry{LSST Project Office} {name={LSST Project Office}, description={Official name of the stand-alone AURA operating center responsible for execution of the LSST construction project under the NSF MREFC account}}
\newglossaryentry{LSST Science Pipelines} {name={LSST Science Pipelines}, description={software used to perform the LSST data reduction pipelines.lsst.io}}
\newacronym{LSSTC} {LSSTC} {\gls{LSST Corporation}}
\newacronym{LSSTPO} {LSSTPO} {\gls{LSST Project Office}}
\newacronym{MPO} {MPO} {Memorandum Purchase Order}
\newacronym{MREFC} {MREFC} {\gls{Major Research Equipment and Facility Construction}}
\newglossaryentry{Major Research Equipment and Facility Construction} {name={Major Research Equipment and Facility Construction}, description={the NSF account through which large facilities construction projects such as LSST are funded}}
\newacronym{NICRA} {NICRA} {(federally) Negotiated Indirect Cost Rate Agreement}
\newacronym{NOIRLab} {NOIRLab} {NSF's National Optical-Infrared Astronomy Research Laboratory; \url{https://nationalastro.org}}
\newacronym{NSF} {NSF} {\gls{National Science Foundation}}
\newglossaryentry{National Science Foundation} {name={National Science Foundation}, description={primary federal agency supporting research in all fields of fundamental science and engineering; NSF selects and funds projects through competitive, merit-based review}}
\newglossaryentry{Operations} {name={Operations}, description={The 10-year period following construction and commissioning during which the LSST Observatory conducts its survey}}
\newacronym{PCW} {PCW} {Project Community Workshop}
\newacronym{POP} {POP} {Project Operating Plan}
\newacronym{PSF} {PSF} {Point Spread Function}
\newglossaryentry{Project Manager} {name={Project Manager}, description={The person responsible for exercising leadership and oversight over the entire Rubin project; he or she controls schedule, budget, and all contingency funds}}
\newacronym{Q2} {Q2} {Quarter two}
\newacronym{QA} {QA} {\gls{Quality Assurance}}
\newacronym{QC} {QC} {\gls{Quality Control}}
\newglossaryentry{Quality Assurance} {name={Quality Assurance}, description={All activities, deliverables, services, documents, procedures or artifacts which are designed to ensure the quality of DM deliverables. This may include QC systems, in so far as they are covered in the charge described in LDM-622. Note that contrasts with the LDM-522 definition of “QA” as “Quality Analysis”, a manual process which occurs only during commissioning and operations. See also: Quality Control}}
\newglossaryentry{Quality Control} {name={Quality Control}, description={Services and processes which are aimed at measuring and monitoring a system to verify and characterize its performance (as in LDM-522). Quality Control systems run autonomously, only notifying people when an anomaly has been detected. See also Quality Assurance}}
\newacronym{RAS} {RAS} {\gls{Resource Allocation Sheet}}
\newacronym{RBT} {RBT} {Request Beyond Target}
\newacronym{RDM} {RDM} {Rubin \gls{Data Management}}
\newacronym{RDO} {RDO} {Rubin Directors Office}
\newacronym{RDP} {RDP} {Rubin Data Production(Obsolete use \gls{RDM})}
\newacronym{REO} {REO} {Rubin Education and Outreach}
\newacronym{RFC} {RFC} {Request For Comment}
\newacronym{ROO} {ROO} {Rubin Observatory \gls{Operations}}
\newacronym{RPF} {RPF} {Rubin system PerFormance}
\newacronym{RSS} {RSS} {square root of the sum of the squares}
\newacronym{RTN} {RTN} {Rubin Technical Note}
\newglossaryentry{Release} {name={Release}, description={Publication of a new version of a document, software, or data product. Depending on context, releases may require approval from Project- or DM-level change control boards, and then form part of the formal project baseline}}
\newglossaryentry{Resource Allocation Sheet} {name={Resource Allocation Sheet}, description={Shows the detailed FTE loading to produce NOIRLab budgets}}
\newglossaryentry{Review} {name={Review}, description={Programmatic and/or technical audits of a given component of the project, where a preferably independent committee advises further project decisions, based on the current status and their evaluation of it. The reviews assess technical performance and maturity, as well as the compliance of the design and end product with the stated requirements and interfaces}}
\newglossaryentry{Risk} {name={Risk}, description={The degree of exposure to an event that might happen to the detriment of a program, project, or other activity. It is described by a combination of the probability that the risk event will occur and the consequence of the extent of loss from the occurrence, or impact. Risk is an inherent part of all activities, whether the activity is simple and small, or large and complex}}
\newglossaryentry{Rubin Operations} {name={Rubin Operations}, description={operations phase of Vera C. Rubin Observatory}}
\newacronym{SLAC} {SLAC} {\gls{SLAC National Accelerator Laboratory}}
\newglossaryentry{SLAC National Accelerator Laboratory} {name={SLAC National Accelerator Laboratory}, description={ A national laboratory funded by the US Department of Energy (DOE); SLAC leads a consortium of DOE laboratories that has assumed responsibility for providing the LSST camera. Although the Camera project manages its own schedule and budget, including contingency, the Camera team’s schedule and requirements are integrated with the larger Project.  The camera effort is accountable to the LSSTPO.}}
\newacronym{SP} {SP} {System PerFormance}
\newglossaryentry{Safety} {name={Safety}, description={The control of accidental loss}}
\newglossaryentry{Science Pipelines} {name={Science Pipelines}, description={The library of software components and the algorithms and processing pipelines assembled from them that are being developed by DM to generate science-ready data products from LSST images. The Pipelines may be executed at scale as part of LSST Prompt or Data Release processing, or pieces of them may be used in a standalone mode or executed through the Rubin Science Platform. The Science Pipelines are one component of the LSST Software Stack}}
\newglossaryentry{Science Platform} {name={Science Platform}, description={A set of integrated web applications and services deployed at the LSST Data Access Centers (DACs) through which the scientific community will access, visualize, and perform next-to-the-data analysis of the LSST data products}}
\newglossaryentry{Software Stack} {name={Software Stack}, description={Often referred to as the LSST Stack, or just The Stack, it is the collection of software written by the LSST Data Management Team to process, generate, and serve LSST images, transient alerts, and catalogs. The Stack includes the LSST Science Pipelines, as well as packages upon which the DM software depends. It is open source and publicly available}}
\newglossaryentry{Subsystem} {name={Subsystem}, description={A set of elements comprising a system within the larger LSST system that is responsible for a key technical deliverable of the project}}
\newglossaryentry{Subsystem Manager} {name={Subsystem Manager}, description={responsible manager for an LSST subsystem; he or she exercises authority, within prescribed limits and under scrutiny of the Project Manager, over the relevant subsystem's cost, schedule, and work plans}}
\newglossaryentry{Summit} {name={Summit}, description={The site on the Cerro Pach\'{o}n, Chile mountaintop where the LSST observatory, support facilities, and infrastructure will be built}}
\newglossaryentry{Systems Engineering} {name={Systems Engineering}, description={an interdisciplinary field of engineering that focuses on how to design and manage complex engineering systems over their life cycles. Issues such as requirements engineering, reliability, logistics, coordination of different teams, testing and evaluation, maintainability and many other disciplines necessary for successful system development, design, implementation, and ultimate decommission become more difficult when dealing with large or complex projects. Systems engineering deals with work-processes, optimization methods, and risk management tools in such projects. It overlaps technical and human-centered disciplines such as industrial engineering, control engineering, software engineering, organizational studies, and project management. Systems engineering ensures that all likely aspects of a project or system are considered, and integrated into a whole}}
\newacronym{T/CAM} {T/CAM} {Technical/Control (or Cost) Account Manager}
\newacronym{UK} {UK} {United Kingdom}
\newacronym{US} {US} {United States}
\newacronym{VRO} {VRO} {(not to be used)Vera C. Rubin Observatory}
\newglossaryentry{Validation} {name={Validation}, description={A process of confirming that the delivered system will provide its desired functionality; overall, a validation process includes the evaluation, integration, and test activities carried out at the system level to ensure that the final developed system satisfies the intent and performance of that system in operations}}
\newglossaryentry{Verification} {name={Verification}, description={The process of evaluating the design, including hardware and software - to ensure the requirements have been met;  verification (of requirements) is performed by test, analysis, inspection, and/or demonstration}}
\newacronym{WBS} {WBS} {\gls{Work Breakdown Structure}}
\newglossaryentry{Work Breakdown Structure} {name={Work Breakdown Structure}, description={a tool that defines and organizes the LSST project's total work scope through the enumeration and grouping of the project's discrete work elements}}
\newglossaryentry{airmass} {name={airmass}, description={The pathlength of light from an astrophysical source through the Earth's atmosphere. It is given approximately by sec z, where z is the angular distance from the zenith (the point directly overhead, where airmass = 1.0) to the source}}
\newglossaryentry{algorithm} {name={algorithm}, description={A computational implementation of a calculation or some method of processing}}
\newglossaryentry{astronomical object} {name={astronomical object}, description={A star, galaxy, asteroid, or other physical object of astronomical interest. Beware: in non-LSST usage, these are often known as sources}}
\newglossaryentry{cadence} {name={cadence}, description={The sequence of pointings, visit exposures, and exposure durations performed over the course of a survey}}
\newglossaryentry{calibration} {name={calibration}, description={The process of translating signals produced by a measuring instrument such as a telescope and camera into physical units such as flux, which are used for scientific analysis. Calibration removes most of the contributions to the signal from environmental and instrumental factors, such that only the astronomical component remains}}
\newglossaryentry{camera} {name={camera}, description={An imaging device mounted at a telescope focal plane, composed of optics, a shutter, a set of filters, and one or more sensors arranged in a focal plane array}}
\newglossaryentry{cycle} {name={cycle}, description={The time period over which detailed, short-term plans are defined and executed. Normally, cycles run for six months, and culminate in a new release of the LSST Software Stack, however this need not always be the case}}
\newglossaryentry{element} {name={element}, description={A node in the hierarchical project WBS}}
\newglossaryentry{epic} {name={epic}, description={A self contained work with a concrete deliverable which my be scheduled to take place with a single cycle and WBS element}}
\newglossaryentry{flux} {name={flux}, description={Shorthand for radiative flux, it is a measure of the transport of radiant energy per unit area per unit time. In astronomy this is usually expressed in cgs units: erg/cm2/s}}
\newglossaryentry{middleware} {name={middleware}, description={Software that acts as a bridge between other systems or software usually a database or network. Specifically in the Data Management System this refers to Butler for data access and Workflow management for distributed processing.}}
\newglossaryentry{monitoring} {name={monitoring}, description={In DM QA, this refers to the process of collecting, storing, aggregating and visualizing metrics}}
\newglossaryentry{passband} {name={passband}, description={The window of wavelength or the energy range admitted by an optical system; specifically the transmission as a function of wavelength or energy. Typically the passband is limited by a filter. The width of the passband may be characterized in a variety of ways, including the width of the half-power points of the transmission curve, or by the equivalent width of a filter with 100\% transmission within the passband, and zero elsewhere}}
\newglossaryentry{seeing} {name={seeing}, description={An astronomical term for characterizing the stability of the atmosphere, as measured by the width of the point-spread function on images. The PSF width is also affected by a number of other factors, including the airmass, passband, and the telescope and camera optics}}
\newglossaryentry{software} {name={software}, description={The programs and other operating information used by a computer.}}
\newglossaryentry{story} {name={story}, description={A JIRA issue type describing a scheduled, self-contained task worked as part of an epic.  Typically, stories are appropriate for work worth between a fraction of a SP and 10 SP; beyond that, the work is insufficiently fine-grained to schedule as a story.  While fractional SP are fine, all stories involve work, so the SP total of an in progress or completed story should not be 0}}
\newglossaryentry{timebox} {name={timebox}, description={A limited time period assigned to a piece of work or other activity.  Useful in scheduling work which is not otherwise easily limited in scope, for example research projects or servicing user requests}}
\newglossaryentry{transient} {name={transient}, description={A transient source is one that has been detected on a difference image, but has not been associated with either an astronomical object or a solar system body}}

\makeglossaries

\title{Work Management Systems for Rubin Operations}

% Optional subtitle
% \setDocSubtitle{A subtitle}

\author{%
William O'Mullane, Amanda Bauer, Robert Blum, Phil Marshall, Cathy Petry
}

\setDocRef{RTN-005}
\setDocUpstreamLocation{\url{https://github.com/rubin-observatory/rtn-005}}

\date{\vcsDate}

% Optional: name of the document's curator
% \setDocCurator{The Curator of this Document}

\setDocAbstract{%
This document describes guidelines for the management of effort in Vera C. Rubin Observatory Operations. It describes the process for planning and executing agile-based work, including tasks that are carried out with regularity (nightly or monthly, etc), long-term development work that iteratively incorporates user feedback, and level of effort activity. The process for measuring progress towards an annually-planned schedule is described as well as the budget planning cycle and the tools used in this process.
}

% Change history defined here.
% Order: oldest first.
% Fields: VERSION, DATE, DESCRIPTION, OWNER NAME.
% See LPM-51 for version number policy.
\setDocChangeRecord{%
\addtohist{1}{2021-11-05}{Unreleased.}{William O'Mullane}
\addtohist{2}{2023-01-13}{Updated with FY23 processes.}{Cathy Petry}
}


\begin{document}

% Create the title page.
\maketitle
% Frequently for a technote we do not want a title page  uncomment this to remove the title page and changelog.
% use \mkshorttitle to remove the extra pages

\section{Introduction}

% Here's a suggestion 

This document provides a guide to the \VRO\ approach to work management and annual planning.
See the operations proposal \citeds{RDO-018} for a description of the full scope and high level goals of the program.
There is no formal \gls{EVMS} required from the funding Agencies (the National Science Foundation (NSF) and the Department of Energy (DOE) Office of Science) so a schedule is planned in detail on an annual basis and effort towards that schedule is tracked through an agile process.

The annual planning process starts by setting high level Milestones for the year, which are centered around releasing data to the public and major maintenance to the telescope system once we enter the phase of full survey operations.   
The Leadership Team builds a series of milestone activities that are discrete pieces of work within Departments and Teams to collectively deliver the high level milestones.  
Teams record their day to day work in Jira and overall progress is monitored automatically through Smartsheet and reported to the Agencies through our managing organizations.

This framework allows the multidisciplinary Rubin teams to operate the facility and generate nightly data products while continuously improving efficiency of workflows,
as well as iteratively responding to user community feedback on a longer timescale to maximize the scientific benefit of annual data releases.
Examples include optimizing the observing strategy as the survey progresses, improving algorithms in response to the user community feedback, 
and other incremental work needed to produce the annual data releases.

In this document, we lay out the procedural details for how we define and carry out annual plans, effectively track work progress to ensure delivery of milestones, maintain visibility in our workflows, remain responsive to change, and offer staff the ability to innovate and collaborate.

\section{Rubin Operations Leadership and Annual Reporting}
\label{sec:contacts}

Rubin Observatory is a Program of NSF's NOIRLab.
The Rubin Observatory \ \gls{Director} is Robert Blum, the Deputy \gls{Director} for NOIRLab is Amanda Bauer, and the Deputy \gls{Director} for SLAC is Phil Marshall.
They are the first point of contact for all issues regarding project management within \RO Operations.

The Program Coordinator is Cathy Petry.
She monitors the budgets and maintains details within the NOIRLab budgeting system.
She assists in developing the annual Program Operating Plan (POP), tracking milestones and reporting on progress.

On the SLAC side, Christine Soldahl is the Business Manager who handles similar tasks.

The POP is a defined report and process for NOIRLab.
Rubin Operations considers the POP to be a Rubin activity, which informs both NOIRLab and SLAC leadership of the annual Rubin activity including milestones and budget.
For NOIRLab, the Rubin POP is integrated and delivered to NSF for the next fiscal year at the end of the current fiscal year.
For SLAC, the POP informs SLAC's annual planning, which culminates in a Field Work Proposal (FWP) for all SLAC High Energy Physics activity including Rubin.

The FWP is delivered in June of the current fiscal year, and covers the federal budget request for the next two fiscal years.
The FWP is previewed to SLAC management and DOE in February in advance of the final delivery in June.
Because the POP for NSF and FWP for DOE are out of phase, Rubin does high level planning in early Q2 of the financial year (January and February).
Detailed activity planning, including defining smaller chunks of work as lower level milestones, continues though the year in advance of the next year.
This detailed activity planning is the subject of this document.

\section{Formal Organizational Structure}
\label{sec:structure}

Rubin Operations has four operational Departments in addition to the Director’s Office: RDO (Rubin Director’s Office), ROO (Rubin Observatory Operations), RDP (Rubin Data Production), RPF (Rubin System Performance), and REO (Rubin Education and Public Outreach).
Each operational Department is led by an Associate Director (AD).

\subsection{Work Breakdown Structure}
\label{sec:wbs}

The \gls{WBS} provides a hierarchical description of the activity-based organization of Rubin.
The WBS provides a useful structure to organize Rubin Operations and plan annual work around.
Rubin is the level 1 of the WBS, the departments are level 2, and teams within the departments are level 3.
Individual roles on operations are level 4.

Thus:

% PJM: The WBS Activities table down to level 3 is a very big table. Do we need the table below to name the dept/team/group as well as the description of the activity? I think we can introduce these items in teh table even though they are not referred to until the next subsection.
\begin{longtable}[]{@{}lll@{}}
\hline
\gls{WBS} & Description \tabularnewline
\hline
\endhead

1 & Rubin Director's Office  \tabularnewline
2 & Rubin Observatory Operations  \tabularnewline
3 & Rubin Data Production \tabularnewline
4 & Rubin System Performance  \tabularnewline
5 & Rubin Education and Public Outreach \tabularnewline

\hline
\end{longtable}

%\subsection{The Control Account Manager}
%\label{sec:cam}

Detailed work is planned in advance of each financial year at the team level.
Team leads will work with department associate directors to develop plans for activities that address specific milestones, projects, and level of effort activity.
The highest level milestones are reported regularly throughout the fiscal year to SLAC, AURA, NSF and DOE.


\subsection{Level of Effort Work}
\label{sec:loe}

There is work throughout Rubin Operations that is recorded as a \gls{LOE}.
These activities include attending meetings, reporting on milestones, or taking part in other activities which do not directly map to deployed code.
This may be particularly the case for technical managers or others in leadership roles.
\gls{LOE} work is assumed to earn value (informally) simply through the passage of time.

In general, we strive to minimize the fraction of effort which is devoted to \gls{LOE} activities and favor those which are more directly accountable.
However, in certain cases such as operations and maintenance of telescope and facility systems, pipelines or other systems, \gls{LOE} is perfectly acceptable.

As an example, a first-order estimate is that developers will spent 30\% of their time on \gls{LOE} type activities, and the remaining 70\% of their effort is tracked against concrete deliverables.
Technical staff in Chile at the summit facility may spend a much more significant fraction of time as  \gls{LOE}.
However, as above, we generally aspire to minimize the fraction of \gls{LOE} for development activity.


\section{Estimating Effort}
\label{sec:effort}

\subsection{Basic Assumptions}
Rubin Operations assumes that a full-time individual works for a total of 1,800 hours per year: this figure is \emph{after} all vacations, sick leave, etc are taken into account.
The Rubin operations partners, SLAC and NOIRLab, may have different definitions for tracking their staff time; Rubin operations uses 1,800 hours per year as a fiducial value for effort estimation purposes.

In general, staff in Rubin operations roles at a given expected full-time equivalent (FTE) effort level are expected to devote that fraction of their total work time to \RO.

Staff in ``scientist'' roles are expected to spend 20\% of their time on personal research (see the Rubin Operations Plan for details).
That is, scientists are expected to devote 1,440 hours per
year to operations activity, and the remainder of their time to personal research.

Personal research time is charged to Rubin along with the operations role time, the logic being that
scientists in operations roles must be engaged in research as well in order to succeed in their operations role.
Rubin expects to pay the full rate for an individual with research time who contributes 1,440 hours to
operations.
This is handled through indirect rates at both NOIRLab and direct charges to research accounts at SLAC.
Science time is included in the subcontracts of our partners at affiliated institutions through indirect charges
similar to the case for NOIRLab.

In Data Production, the base assumption is that 30\% of an individual's \RO operations time (i.e. 540 hours/year for a full-time developer, 432 hours/year for a full-time scientist) are devoted to
overhead for regular meetings\footnote{``Meetings'' include, for example, scheduled weekly team meetings, stand-ups, etc;
major conferences or project meetings involving preparation, travel time, etc should be scheduled in advance and allocated \glspl{SP}.},
ad-hoc discussions and other interruptions.
This work is counted as \gls{LOE}.
It is actively encouraged to allocate less than 30\% of an individuals time to \gls{LOE} where that is possible.

Assuming no variation throughout the year, we therefore expect 105 hours of productive work from a developer, or 84 hours from a scientist, per month.
Note that this is averaged across the year: some months, such as those containing major holidays, will naturally involve less working time than others: the remainder will necessarily include more working time to compensate. For other staff, the \gls{LOE} will be higher but include much more day to day activity than for the developer case.

Rather than working in hours, our Jira based system uses Story Points (\gls{SP}), with one \gls{SP} being defined as equivalent to four hours of effort (half a day's work) by a competent developer.

Thus, we expect developers and scientists to produce 26.25 and 21 \glspl{SP} per \emph{average} month respectively.

\begin{table}
\begin{longtable}[]{@{}lrrr@{}}
\hline
          & \multicolumn{2}{c}{Hours} & \multicolumn{1}{c}{\glspl{SP}} \\
          & Per year & Per month      & Per month \\
\hline
Full-time Developer & 1800     & 105            & 26.25 \\
Full-time Scientist & 1440     &  84            & 21.00 \\
\hline
\end{longtable}
\caption{Expected working rates for developers and scientists. Technicians and engineers follow the same rates as developers.}
\label{tab:working-rate}
\end{table}

\subsection{Special Cases}

\subsubsection{Newcomers}
\label{sec:newcomers}

New or inexperienced developers, even when devoting their full attention to \gls{story}-pointed work, will likely be less productive than their more experienced peers.
In this case, the ratio of hours to \glspl{SP} increases, but the number of hours remains constant.

Note that specific activities related on ``onboarding'' and getting up to speed with operations can be ticketed as regular work.
For example, working through tutorials, reading documentation, and so on are all activities which can earn \glspl{SP}.

\subsubsection{Technical Managers and other Leadership Roles}

Individuals in leadership roles may find it necessary to assign a larger fraction of their time to \gls{LOE} type work, and therefore spend fewer hours generating \glspl{SP}.
The ratio of hours to \glspl{SP} remains constant, but the number of hours decreases.


% should be explain the "annual planning" process here and emphasize what the Agencies (NSF/NOIRLab) are asking us to report?  Milestones, Milestone Activities, Deliverables.  I could imagine have a brief section describing annual planning, a section on progress tracking, a section on reporting.  We should also comment on the difference between the DOE and NSF side regarding FWP and annual planning timing.  It works in our favor, but we could summarize that in the way Phil did in his JODR presentation.
\section{Long Term Planning}
\label{sec:long-term-plan}

The authoritative, high-level summary of the long-term planning system may be found in any POP process document.

Here we expand upon the details of that system.
The plan for pre-operations and Survey Operations is embodied in:

\begin{enumerate}
    \item A set of \emph{Milestones}, each of which represents the delivery of a major aspect of Rubin Operations, availability of specific functionality, or maintenance event for the telescope system.
        Milestones are planned in Smartsheet and then officially defined in a \gls{JIRA} milestone issue.
    \item A series of \emph{epics} describe major pieces of work.
        Epics are associated with concrete, albeit high-level, deliverables or outcomes that culminate in the achievement of the above milestones, and have specific resource loads (staff assignments story point values) and end dates.
        All epics are linked to the milestone they are created to help deliver, although some epics might exist without linking to a milestone (level of effort or emergent work epics, for example).
    \item A visualization of progress on work done towards achieving milestones is captured in Smartsheet, which directly tracks progress by rolling up issues that are completed inside of \gls{JIRA} epics that work together to deliver a given milestone.
\end{enumerate}

Milestones are allocated to one of three levels, defined as follows:

\begin{description}
\item[Level 1] These are at the full observatory level and are owned by the Directors Office.
Examples are the completion of a Data Preview, the beginning of nightly observations for the full survey, or the delivery of an annual Data Release.
Level 1 milestones are achieved by the culmination of effort defined by a set of Level 2 and Level 3 milestones.
Level 1 milestones are reported to the agencies.
\item[Level 2] These reflect effort within a Department and are owned by an Associate Director, or are cross-Department commitments.
As such, they must be defined in consultation with the Director's Office.
Level 2 milestones are achieved by the culmination of effort defined by a set of Level 3 milestones.
Some Level 2 milestones {\it may} be reported to the agencies as defined by the annual POP.
\item[Level 3] These are internal to a particular Department and assigned to a team and can therefore be specified by a single team lead.
\end{description}

Some of these milestones are exposed to external reviewers (usually reserved for Level 1 milestones): it is important that these be delivered on time and to specification.
Level 1 and 2 milestones are under change control once they are defined and described in a \gls{JIRA} Milestone issue.
Level 3 milestones are defined for use within Departments and not required to go under project change control, but properly adhering to the plan is important: your colleagues in other teams will use these milestones to align their schedules with yours, so they rely on you to be accurate.

Epics should work to achieve milestones i.e. they may be blocking issues on the milestones.
When a detailed description of work for a given \gls{epic} is known, it is described in \gls{JIRA}.
It should then be assigned to the appropriate \glspl{cycle}.

Progress is tracked toward achieving milestones in Smartsheet by monitoring completed story points in \gls{JIRA} epics and rolling up the total progress.
To ensure success, \gls{JIRA} epics must be completely detailed out prior to a full 6-month cycle and total effort should be estimated out for an entire fiscal year of effort, as detailed below.

\subsection{Planning Research Work}
\label{sec:long-term-research}

In order for \RO  to reach its science goals, new algorithmic or engineering approaches must sometimes be researched.
It is appropriate to budget time for this research work in planning packages.


\subsection{Epic-Based Long Term Plans}

As long as they have not been scheduled for the current \gls{cycle}, these \glspl{epic} can be freely created and changed at any time, without any sort of approval process.

Fine grained planning of this sort can be useful for ``bottom-up''
analysis of the work to be performed and validation of the resources
needed to implement a particular planning package. Thinking through the
plan in this way can help in building up a detailed plan in a flexible,
agile way, while also ensuring that scope, cost and schedule are
carefully controlled.


\subsection{Defining the Schedule with Milestones}

Rubin Milestones are defined as Jira issues of type ``Milestone''. As indicated above, the Director (or their designate) defines the L1 milestones, the Associate Directors (ADs) define their departments' L2 milestones, and the Team Leaders and ADs define the L3 milestones for their teams.

L1, L2 and most L3 milestones are defined as part of the annual planning cycle, and prior to the year in which the work associated with them is due to be carried out. ADs and Team Leaders communicate their milestones to the Program Coordinator, who enters them into Smartsheet and then creates a Jira issue of type ``milestone'' for each one.

During the year, it is sometimes necessary to create new milestones (primarily at Level 3) that were omitted during the earlier planning phase. In this case, the team leader or AD may create the Jira milestone directly, and alert the Program Coordinator to it for inclusion in the Smartsheet.

The following fields must be filled out when defining a new milestone:
\begin{itemize}
\item The \texttt{Milestone Level}, ``1,'' ``2,'' or ``3;''
\item The \texttt{RO Milestone ID}.
  This is a string like ``L2-DI-0003'' which indicates the level of the milestone, the department (for level 2 milestones) or team (for level 3 milestones) that owns it, and an ID number that serves to make the ID unique wbut which otherwise carries no meaning.
  The list of two-character department and team string identifiers can be found in the Appendix (``DI'' in the example stands for ``Directorate'').
  The ID number may get edited by the Director's Office to ensure that it is unique: the Milestone IDs are only ever used in linking together the milestones for visualization in Smartsheet.
  Good practice here is to look at the milestone list in Smartsheet and choose an ID number that leads to a milestone ID that has not already been taken, and which roughly fits the milestone into the time-ordered list.
\item The \texttt{Summary} field should contain a sentence outlining either what needs to be done, or what will be delivered for the Milestone to be reached.
  Examples include \textit{``Deliver Data Preview 0.1 (DP0.1)''} (an L1 milestone) and \textit{``DP0.1 Data Release: science-ready catalogs released from the IDF''} (an L2 milestone that belongs to it).
\item The \texttt{Activity Description} text should contain one or maybe two sentences summarizing the activity needed to complete the milestone.
  Example: \textit{``Upgrade the WBS activity, labor and non-labor plans from V4 to V5 in order to capture a US DF at SLAC, a UK DF, and any other modifications needed, and estimate the corresponding budget.''}
  Note that only L1 (and sometimes L2) milestones are actually listed in the NOIRLab Program Operations Plan (POP), but Rubin adopts the same structure for all its milestones.
\item The \texttt{Deliverable Description} is a very terse list of the deliverables needed to reach the Milestone.
  Example: \textit{``V5 WBS workbook and Preliminary Cost Calculator.''}
\item The \texttt{Due Date} is the latest date in the future by which the milestone needs to be reached.
  This date should be before or the same as the milestone's parent milestone's ``Rubin Forecast Due Date'' as shown in the Smartsheet.
\item The \texttt{Start Date} is the date when the work for the milestone should begin.
  This is the date that the Smartsheet will use in a visual comparison between the fraction of work completed and the fraction of time elapsed, to help track progress on the epics.
\end{itemize}

The Program Coordinators will ensure that the milestones that have been defined are correctly linked together in the Smartsheet, so that their epics appear nested beneath them.


\section{Short Term Planning} \label{sec:cycle-plan}

Short term planning is carried out in blocks referred to as \glspl{cycle}, which (usually) last for six months.
Before the start of a \gls{cycle}, milestones are confirmed by the Director's Office, listed in Smartsheet, and detailed in the Milestone issue. Any team member can find the milestones in Jira.

\subsection{Defining The Plan}

\subsubsection{Scoping Work}

The first essential step of developing the short term plan is to produce an outline of the program of work to be executed.
In general, this should flow directly from the long term plan (\S\ref{sec:long-term-plan}), ensuring that the expected planning packages are being worked on and milestones being hit.

While developing the \gls{cycle}, please:

\begin{itemize}
\item Do not add \emph{artificial} padding or buffers to make the schedule look good;
\item Do budget appropriate time for handling bugs and emergent issues;
\item Reserve time for planning the following \gls{cycle}: it will have to be defined before this \gls{cycle} is complete;
\item Leave time for other necessary activities, such as cross-team collaboration meetings and writing documentation.
\item Per the \gls{cycle} cadence, ensure that new development will conclude (or, at a minimum, be in a releasable state) in time for the end of \gls{cycle} release.
\end{itemize}

Obviously, ensure that the program of work being developed is achievable by your team in the time available: ultimately, you will want to compare the number of \glspl{SP} your team is able to deliver (\S\ref{sec:effort}) with the sum of the \glspl{SP} in the \glspl{epic} you have scheduled (\S\ref{sec:planning-epics}), while also considering the skills and availability of your team.
It is better to under-commit and over-deliver than vice-versa, but, ideally, aim to estimate accurately.

\subsubsection{Defining Epics} \label{sec:planning-epics}

The plan for a six month \gls{cycle} fundamentally consists of a set of resource loaded \glspl{epic} defined in \gls{JIRA}.
Each \gls{epic} loaded into the plan must have this minimum set of fields filled in:

\begin{itemize}
\item A concrete, well defined deliverable \emph{or} be clearly described as a ``bucket'' or ``emergent work'' (\S\ref{sec:bucket});
\item The \texttt{Component} field set to the appropriate Department;
\item The \texttt{Story Points} field set to a (non-zero) estimate of the effort required to complete the \gls{epic} in terms of \glspl{SP} (see \S\ref{sec:effort}).
\item The \texttt{RO Milestone ID} field set to the appropriate L3 or L2 milestone that the epic will contribute to achieving (or achieve in its totality).
  The exact \texttt{RO Milestone ID}text can be found in the Milestone issue or Smartsheet.
\item The \texttt{Due Date} field set to the appropriate date, which does not exceed the due date of the Milestone it is labeled to achieve.
\item The \texttt{label} field is set to identify the fiscal year during which the work will be done. Examples are FY22 or FY23.

\end{itemize}
The fields above are required to have values entered because they define the connection to Smartsheet where effort-tracking for the full project is done. Other fields in the epic can also be filled in as needed.

Be aware that: (Amanda: I don't think all of the following bullets are true or necessary in the Operations non-EVMS era.  Should clarify with Cathy and Wil.)

\begin{itemize}
\item An \gls{epic} may only be assigned to a single \gls{cycle}.
  It is not possible to define an \gls{epic} that crosses the \gls{cycle} boundary (see \S\ref{sec:cycle-close} for the procedure when an \gls{epic} is not complete by the end of the \gls{cycle}).
\item Indeed, where possible management activities \emph{should} be scheduled as \glspl{epic} with concrete deliverables in this \gls{element} rather than being handled as \gls{LOE}.
\item The \gls{epic} should be at an appropriate level of granularity.
  While short \glspl{epic} (a few \glspl{SP}) may be suitable for some activities, in general \glspl{epic} will describe a few months of developer-time.
  \Glspl{epic} allocated multiple hundreds of \gls{story} points are likely too broad to be accurately estimated.
\end{itemize}


Although it is possible---indeed, encouraged---to set the \texttt{assignee} field in \gls{JIRA} to the individual who is expected to carry out the bulk of the work in an \gls{epic}, this does not provide sufficient granularity for those cases when more than one person will be contributing.

In fact, it is only required to provide a staff assignment in terms of ``resource types'' (i.e. scientists, senior scientists, developers, senior developers, etc).
In practice, to ensure your team is evenly loaded, it is usually necessary to break it down to named individuals.


\subsubsection{Scheduling Research Work} \label{sec:research}

%who gets research time of this type? All developers or just scientists. This is confusing since most people will think in terms of scientific research. NOIRLab engineers will get professional development time.
As discussed in \S\ref{sec:long-term-research}, research is sometimes required
to meet our objectives. However, it is not a natural fit to our usual
planning process, as it is speculative in its nature: it is often
impossible to produce a series of logical steps that will lead to the
required result. We acknowledge, therefore, that scheduling an \gls{epic} to
deliver some particular new \gls{algorithm} based on the results of research
is impossible: we cannot predict with any confidence when the
breakthrough will occur.

We therefore schedule research in \gls{timebox}ed \glspl{epic}: we allocate a certain amount of time based on the resources available, rather than on an estimate of time to completion.
However, note that these \gls{timebox}ed \glspl{epic} should still provide concrete deliverables: they are not open-ended ``buckets'' as discussed elsewhere.

\subsubsection{Bucket Epics} \label{sec:bucket}

Some work is ``emergent'': we can predict in advance that it will be necessary, but we cannot predict exactly what form it will take.
The typical example of this is fixing bugs: we can reasonably assume that bugs will be discovered in the codebase and will need to be addressed, but we cannot predict in advance what those bugs will be.

This can be included in the schedule by defining a ``bucket'' \gls{epic} in which stories can be created when necessary during the course of a \gls{cycle}.
Make clear in the description of the \gls{epic} that this is its intended purpose: every \gls{epic} should either have a concrete deliverable or be a bucket.

Bucket \glspl{epic} have some similarities with \gls{LOE} work.
As such, we acknowledge that they are necessary, but seek to minimize the fraction of our resources assigned to them.
If more than a relatively small fraction of the work for a \gls{cycle} is assigned to bucket \glspl{epic}, please consider whether this is really necessary and appropriate.

Be aware that even bucket \glspl{epic} must be assigned to a specific \emph{leaf} \gls{element} of the \gls{WBS}.
That is, it is not in general possible to define an \gls{epic} which handles bug reports or emergent feature requests across the whole of the codebase unless a specific \gls{WBS} leaf \gls{element} is devoted to maintenance activities of this type.
Instead, it may be necessary to define a different bucket \gls{epic} for each leaf of the \gls{WBS} tree.


\subsection{Closing the Cycle} \label{sec:cycle-close}

Assuming everything has gone to plan, by the end of a \gls{cycle} all deliverables should be verified and the corresponding \glspl{epic} should be marked as \texttt{done}.
Marking an \gls{epic} as \texttt{done} asserts that the concrete deliverable associated with the \gls{epic} has been provided.

Epics which are in progress at the end of the \gls{cycle} cannot be closed until they have been completed.
These \glspl{epic} will spill over into the subsequent \gls{cycle}.
It is \emph{not} appropriate to close an in-progress \gls{epic} with a concrete deliverable until that deliverable has been achieved: instead, a variance will be shown until the \gls{epic} can be closed.
Obviously, this will impact the labor available for other activities in the next \gls{cycle}.
(This does not apply to bucket \glspl{epic} (\S\ref{sec:bucket}), which are, by their nature, \gls{timebox}ed within the \gls{cycle}).


\section{Execution} \label{sec:execution}

Having defined the plan for a \gls{cycle} following \S\ref{sec:cycle-plan}, we (RDP and RPF) execute it by means of a series of month-long sprints.
In this section, we detail the procedures teams are expected to follow during the \gls{cycle}.

\subsection{Defining Stories}
\label{sec:defining-stories}

Epics have already been defined as part of the \gls{cycle} plan (see \S\ref{sec:planning-epics}).
However, the \gls{epic} is not at an appropriate level for scheduling day-to-day work.
Rather, each \gls{epic} is broken down into a series of self-contained ``stories''.
A \gls{story} describes a planned activity worth between a small fraction of a SP and several \glspl{SP} (more than about 10 is likely an indication that the \gls{story} has not been sufficiently refined).
It must be possible to schedule a \gls{story} within a single sprint, so no \gls{story} should ever be allocated more than 26 \glspl{SP}.

The process for breaking \glspl{epic} down into stories is not mandated. In
some circumstances, it may be appropriate for the technical manager to
provide a breakdown; in others, they may request input from the
developer who is actually going to be doing the work, or even hold a
brainstorming session involving the wider team. This is a management
decision.

It is not required to break all \glspl{epic} down into stories before the \gls{cycle} begins: it may be more appropriate to first schedule a few exploratory stories and use them to inform the development of the rest of the \gls{epic}.
However, do break \glspl{epic} down to describe the stories which will be worked in an upcoming sprint (\S\ref{sec:sprinting}) before the sprint starts.
When doing so, you may wish to leave some spare time to handle emergent work (discussed in \S\ref{sec:bugs}).

Note that there is no relationship enforced between the \gls{SP} total estimated for the \gls{epic} and the sum of the \glspl{SP} of its constituent stories.
It is therefore possible to over- or under-load an \gls{epic}.
This will have obvious ramifications for the schedule.

\subsection{Sprinting}
\label{sec:sprinting}

Each team organizes its work around periods of work called sprints.
A sprint comprises a defined collection of stories which will be addressed over the course of the month.
These stories are not necessarily (indeed, not generally) all drawn from the same \gls{epic}: rather, while \glspl{epic} divide the \gls{cycle} along logical grounds, sprints divide it along the time axes.

Broadly, executing a sprint falls into three stages:

\begin{enumerate}
\item Preparation.

  The team assigns the work that will be addressed during the sprint by choosing from the pre-defined stories (\S\ref{sec:defining-stories}).
  Each team member should be assigned a plausible amount of work, based on the per-\gls{story} \gls{SP} estimates and the likely working rate of the developer (see \S\ref{sec:effort}).

  The process by which work is assigned to team members is a local
  management decision: the orthodox approach is to call a team-wide
  meeting and discuss it, but other approaches are possible (one-to-one
  interactions between developers and technical manager, managerial
  fiat, etc).

  Do not overload developers. Take vacations and holidays into account.
  The sprint should describe a plausible amount of work for the time
  available.
\item Execution.

  Daily management during the sprint is a local decision. Suggested best
  practice includes holding regular ``standup'' meetings, at which
  developers discuss their current activities and try to resolve
  ``blockers'' which are preventing them from making progress.

  Stories should be executed following the instructions in the
  \href{http://developer.lsst.io/}{Developer Guide} as regards workflow,
  coding standards, review requirements, and so on. It is important to
  ensure that completed stories are marked as \texttt{done}:
  experience suggests that this can easily be forgotten as developers
  rush on to the next challenge, but it is required to enable us to
  properly track earned value as per \S\ref{sec:cycle-value}.

  When completing a \gls{story} we do not change the number of \glspl{SP} assigned to
  it: the \gls{SP} total reflects our initial estimate of the work involved,
  not the total time invested.
  However, we should \textit{also} record the true \glspl{SP} expended on the issue.
  This makes it possible to review the quality of our estimates at the end of the sprint.
  Each individual, with guidance from their \gls{T/CAM}, should use this information as they strive to improve the accuracy of their planning and estimating.

  Avoid adding more stories to a sprint in progress unless it is
  unavoidable (for example, the \gls{story} describes a critical bug that must
  be addressed before proceeding). A sprint should always stay current
  and should be up-to-date with reality; if necessary, already scheduled
  stories may be pushed out of a sprint as soon as it is obvious it is
  unrealistic to expect them to be completed.
\item \gls{Review}.

  At the end of the sprint, step back and consider what has been
  achieved. What worked well? What did not? How can these problems be
  avoided for next time? Was your estimate of the amount of work that
  could be finished in the sprint accurate? If not, how can it be
  improved in future? Refer to the
  \href{https://en.wikipedia.org/wiki/Burn_down_chart}{burn-down chart}
  for the sprint, and, if it diverged from the ideal, understand why.

  Again, the form the review takes is a local management decision: it
  may involve all team members, or just a few.
\end{enumerate}

We use \gls{JIRA}'s
\href{https://www.atlassian.com/software/jira/agile}{Agile} capabilities
to manage our sprints. Each technical manager is responsible for
defining and maintaining their own agile board. The board may be
configured for either
\href{https://en.wikipedia.org/wiki/Scrum_(software_development)}{Scrum}
or \href{https://en.wikipedia.org/wiki/Kanban_(development)}{Kanban}
style work as appropriate: the former is suitable for planned
development activities (e.g. \gls{Science Pipelines} development); the latter
for servicing user requests (e.g. providing developer support).

\subsection{Closing Epics}
\label{sec:epic-close}

\subsubsection{Completing the Work}
\label{sec:epic-done}

An \gls{epic} may be marked as \texttt{done} when:

\begin{enumerate}
\item It contains at least one completed \gls{story};
\item There are no more incomplete \glspl{story} defined within it;
\item There are no plans to add more \glspl{story};
\item (If applicable, i.e. it is not a bucket, as defined in \S\ref{sec:bucket}) its concrete deliverable has been achieved.
\end{enumerate}

Note that it is not permitted to close an \gls{epic} without defining at least one \gls{story} within it.
Empty \glspl{epic} can never be completed.

\subsection{Handling Bugs \& Emergent Work}
\label{sec:bugs}

\subsubsection{Receiving Bug Reports}\label{receiving-bug-reports}

Members of the project who have access to \gls{JIRA} may report bugs or make feature requests directly using \gls{JIRA}.
As discussed in \S\ref{sec:jira-maintenance}, technical managers should regularly monitor \gls{JIRA} for relevant tickets and ensure they are handled appropriately.

Our code repositories are exposed to the world in general through \href{https://github.com/lsst/}{GitHub}.
Each repository on GitHub has a bug tracker associated with it.
Members of the public may report issues or make requests on the GitHub trackers.
Per the \href{https://developer.lsst.io/processes/workflow.html}{Developer Workflow}, all new work must be associated with a \gls{JIRA} ticket number before it can be committed to the repository.
It is therefore the responsibility of technical managers to file a \gls{JIRA} ticket corresponding to the GitHub ticket, to keep them synchronized with relevant information, and to ensure that the GitHub ticket is closed when the issue is resolved in \gls{JIRA}.

The GitHub issue trackers are, in some sense, not a core part of our
workflow, but they are fundamental to community expectations of how they
can interact with the project. Ensure that issues reported on GitHub are
serviced promptly.

In some cases, the technical manager responsible for a given repository
is obvious, and they can be expected to take the lead on handling
tickets. Often, this is not the case: repositories regularly span team
boundaries. Work together to ensure that all tickets are handled.

\subsubsection{Issue Types}\label{issue-types}


\subsection{Jira Maintenance}
\label{sec:jira-maintenance}

At any time, new tickets may be added to \gls{JIRA} by team members.
Please remind your team of the best practice in this respect (\jira{RFC-147}).
It is the responsibility of technical managers to ensure that new tickets are handled appropriately, updating the schedule to include them where necessary.

It is required that the \texttt{Team} field be set to the appropriate team (\jira{RFC-145}).
This indicates which manager is responsible for \gls{seeing} that the work is completed successfully.
Available teams, and the associated managers, are listed in the \href{https://developer.lsst.io/processes/jira_agile.html}{Developer Guide}; generally speaking, they align with the the work breakdown structure described in \S\ref{sec:wbs}.
Where there is uncertainty about which team should be responsible for a particular ticket, the “System Management” team may be used to indicate that the \gls{DM} \gls{Project Manager} is responsible for assigning the work.

Please regularly monitor \gls{JIRA} for incomplete tickets and update them appropriately.
Where tickets describe bugs or other urgent emergent work which cannot be deferred, refer to \S\ref{sec:bugs}.

\subsection{Coordination Standup}
\label{sec:sup}

\section{Tracking Progress and Standard Reporting Cycle}
\label{sec:reporting-cycle}
 Cathy ?

\subsection{Tracking Progress toward Milestones}

Progress on completing epics is visualized in Smartsheet.
Smartsheet lists all Level 1 through Level 3 milestones in a gantt-chart style view.
Each Level 1 milestone is achieved by completing a series of Level 2 and/or Level 3 milestones.
Smartsheet tracks Story Points marked as complete in individual \gls{JIRA} epics in real time.
Progress on individual milestones is shown as the weighted total of Story Points within each epic contributing to the successful completion of the milestone.

\subsection{Reporting Cycle}

High level milestone progress will be reported to SLAC and NOIRLab regularly.
NOIRLab reports will flow quarterly (or monthly) to the NSF.
Rubin will show progress on all L1 milestones and any L2 milestones called out in the POP.


\section{Personnel}

\subsection{Staffing Changes}
\label{sec:staffing}

In addition to onboarding procedures at your local institution, please
be aware of

\begin{itemize}
\item The \gls{LSST} \href{https://project.lsst.org/onboarding}{New Employee
  Onboarding} material, and
\end{itemize}

and direct new recruits to them when they join your team\footnote{As per \S\ref{sec:newcomers}, remember that newcomers should be allocated \glspl{SP} for working through this material.}.

The responsible hirere must also complete an \href{https://project.lsst.org/onboarding/form}{onboarding form} for the new recruit.
When members of staff team leave the project, the \gls{T/CAM} should fill in an \href{https://project.lsst.org/onboarding/offboarding_form}{offboarding form}.


\section{Open issues}

\begin{itemize}
\item Kanban for \gls{LOE} operations work
\item Need section on more procedural driven work on mountain and DF.
\end{itemize}


\appendix
% Include all the relevant bib files.
% https://lsst-texmf.lsst.io/lsstdoc.html#bibliographies
\section{References} \label{sec:bib}
\renewcommand{\refname}{} % Suppress default Bibliography section
\bibliography{local,lsst,lsst-dm,refs_ads,refs,books}

% Make sure lsst-texmf/bin/generateAcronyms.py is in your path
%\section{Acronyms} \label{sec:acronyms}
%\input{acronyms.tex}
\printglossaries





\end{document}
